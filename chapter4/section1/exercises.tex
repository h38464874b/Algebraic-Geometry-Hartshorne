\documentclass{article}
\usepackage[margin=1in]{geometry}
\usepackage{amsmath,amsfonts,mathtools,amsthm,amssymb}
\usepackage[l2tabu,orthodox]{nag}
\usepackage{microtype} % fixes spacing or whatever
\usepackage{enumitem}
\usepackage{tikz-cd}
\usepackage{xcolor}
\usepackage{physics}
\usepackage{mathrsfs}
% \usepackage{kpfonts}
% \usepackage[cal=dutchcal,
% bb=boondox,
% bbscaled=1.05,
% scr=boondoxupr]{mathalpha}

\renewcommand{\epsilon}{\varepsilon}
\renewcommand{\phi}{\varphi}

\DeclareMathAlphabet{\mathpzc}{OT1}{pzc}{m}{it}

\newcommand{\goth}[1]{\mathfrak{#1}}
\newcommand{\red}[1]{#1_{\text{red}}}
\newcommand{\reg}[1]{#1_{\text{reg}}}

\newcommand{\fA}{\mathscr{A}}
\newcommand{\fE}{\mathscr{E}}
\newcommand{\fF}{\mathscr{F}}
\newcommand{\fG}{\mathscr{G}}
\newcommand{\fH}{\mathscr{H}}
\newcommand{\fI}{\mathscr{I}}
\newcommand{\fJ}{\mathscr{J}}
\newcommand{\fK}{\mathscr{K}}
\newcommand{\fL}{\mathscr{L}}
\newcommand{\fM}{\mathscr{M}}
\newcommand{\fN}{\mathscr{N}}
\newcommand{\fO}{\mathscr{O}}
\newcommand{\fP}{\mathscr{P}}
\newcommand{\fR}{\mathscr{R}}
\newcommand{\fS}{\mathscr{S}}
\newcommand{\fT}{\mathscr{T}}
\newcommand{\fX}{\mathscr{X}}
\newcommand{\fY}{\mathscr{Y}}
\newcommand{\fZ}{\mathscr{Z}}

\newcommand{\gm}{\goth{m}}
\newcommand{\gF}{\goth{F}}
\newcommand{\gU}{\goth{U}}
\newcommand{\gV}{\goth{V}}

\newcommand{\A}{\mathbf{A}}
\newcommand{\C}{\mathbf{C}}
\newcommand{\F}{\mathbf{F}}
\newcommand{\G}{\mathbf{G}}
\newcommand{\N}{\mathbf{N}}
\newcommand{\PP}{\mathbf{P}}
\newcommand{\Q}{\mathbf{Q}}
\newcommand{\R}{\mathbf{R}}
\newcommand{\Z}{\mathbf{Z}}

\newcommand\srestr[2]{{\left.\kern-\nulldelimiterspace #1\vphantom{\small|} \right|_{#2}}}
\newcommand\restr[2]{{\left.\kern-\nulldelimiterspace #1 \vphantom{\big|} \right|_{#2}}}

\newtheorem{theorem}{Theorem}
\newtheorem{lemma}{Lemma}
\newtheorem{corollary}{Corollary}
\newtheorem{proposition}{Proposition}

\DeclareMathOperator{\rProj}{\mathpzc{Proj}}
\DeclareMathOperator{\rSpec}{\mathpzc{Spec}}
\DeclareMathOperator{\rHom}{\mathpzc{Hom}}
\DeclareMathOperator{\id}{id}
\DeclareMathOperator{\Frac}{Frac}
\DeclareMathOperator{\rk}{rank}
\DeclareMathOperator{\pic}{Pic}
\DeclareMathOperator{\cacl}{CaCl}
\DeclareMathOperator{\trd}{tr.d.}
\DeclareMathOperator{\cl}{Cl}
\DeclareMathOperator{\depth}{depth}
\DeclareMathOperator{\codim}{codim}
\DeclareMathOperator{\Div}{Div}
\DeclareMathOperator{\coker}{coker}
\DeclareMathOperator{\len}{length}
\DeclareMathOperator{\height}{height}
\DeclareMathOperator{\supp}{Supp}
\DeclareMathOperator{\proj}{Proj}
\DeclareMathOperator{\im}{im}
\DeclareMathOperator{\Hom}{Hom}
\DeclareMathOperator{\Der}{Der}
\DeclareMathOperator{\spec}{Spec}
\DeclareMathOperator{\Aut}{Aut}
\DeclareMathOperator{\ch}{char}
\DeclareMathOperator{\tor}{Tor}
\DeclareMathOperator{\Ann}{Ann}
\DeclareMathOperator{\Syl}{Syl}
\DeclareMathOperator{\Sym}{Sym}
\DeclareMathOperator{\GL}{GL}
\DeclareMathOperator{\SL}{SL}
\DeclareMathOperator{\Stab}{Stab}
\DeclareMathOperator{\Perm}{Perm}
\DeclareMathOperator{\Orb}{Orb}
\DeclareMathOperator{\Gal}{Gal}
\DeclareMathOperator{\Supp}{Supp}

\author{James Lee}

% Solarized
% \pagecolor[RGB]{0,20,26}
% \color[RGB]{191,191,191}

% Sephia
\pagecolor[RGB]{249,239,220}

% Black
% \pagecolor[RGB]{20,20,20}
% \color[RGB]{200,200,200}

\title{Chapter 4, Section 1}

\begin{document}
\maketitle
\begin{enumerate} [label=\textbf{\arabic*.}, leftmargin=0em]

\item[\textbf{1.}] Let $X$ be a curve, and let $P \in X$ be a point.
Then there exists a nonconstant  rational function $f \in K(X)$, which is regular everywhere except at $P$.

\begin{proof}
  Pick another point $Q \neq P \in X$, which exists since we always assume the ground field $k$ is algebraically closed.
  By the Riemann-Roch Theorem, there exists $n > 0 $ such that $\dim|n(-P + 2Q)| > 0$.
  Hence, there exists a rational function with pole at $P$ of order $n$, and regular everywhere else.
\end{proof}

\item[\textbf{2.}] Again let $X$ be a curve, and let $P_1, \dots, P_r \in X$ be points.
Then there is a rational function $f \in K(X)$ having poles (of some order) at each of the $P_i$, and regular elsewhere.

\begin{proof}
  Pick another point $Q \neq P_i \in X$, and repeat (Ex. 1.1) with $n(-\sum P_i + 2rQ)$.
\end{proof}

\item[\textbf{3.}] Let $X$ be an integral, separated, regular, one-dimensional scheme of finite type over $k$, which is \textit{not} proper over $k$.
Then $X$ is affine.

\item[\textbf{4.}] Show that a separated, one-dimensional scheme of finite type over $k$, none of whose irreducible components is proper over $k$, is affine.

\item[\textbf{5.}] For an effective divisor $D$ on a curve $X$ of genus $g$, show that $\dim{|D|} \leq \deg{D}$.
Furthermore, equality holds if and only if $D = 0$ or $g = 0$.

\begin{proof}
  By definition $\dim|D| = \ell(D) - 1$. Rearranging the Riemann-Roch Theorem gives
  \begin{equation*}
    \dim{|D|} = \ell(K - D) + \deg{D} - g,
  \end{equation*}
  so we want to show $\ell(K - D) \leq g$. But $D$ is effective, so $\fL(K - D) \to \fL(K)$ is injective, and $g = \ell(K) = \dim{H^0(X, \fL(K))}$ by definition.
\end{proof}

\item[\textbf{6.}] Let $X$ be a curve of genus $g$.
Show that there is a finite morphism $f : X \to \PP^1$ of degree $\leq g + 1$.

\begin{proof}
  By the Riemann-Roch Theorem, it suffices to show there exists a divisor $D$ such that $2 \leq \ell(D) \leq \ell(K - D)$.
  Indeed, take $D = -P$
\end{proof}

\item[\textbf{7.}] A curve $X$ is called \textit{hyperelliptic} if $g \geq 2$ and there exists a finite morphism $f : X \to \PP^1$ of degree 2.
\begin{itemize}
  \item[(a)] If $X$ is a curve of genus $g = 2$, show that the canonical divisor defines a complete linear system $|K|$ of degree 2 and dimension 1, without base points. Use (II, 7.8.1) to conclude that $X$ is hyperelliptic.
  \item[(b)] Show taht the curves constructed in (1.1.1) all admit a morphism of degree 2 to $\PP^1$. Thus there exist hyperelliptic curves of any genus $g \geq 2$.
\end{itemize}

\end{enumerate}
\end{document}
