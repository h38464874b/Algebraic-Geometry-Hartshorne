\documentclass{article}
\usepackage[margin=1in]{geometry}
\usepackage{amsmath,amsfonts,mathtools,amsthm,amssymb}
\usepackage[l2tabu,orthodox]{nag}
\usepackage{microtype} % fixes spacing or whatever
\usepackage{enumitem}
\usepackage{tikz-cd}
\usepackage{xcolor}
\usepackage{physics}
\usepackage{mathrsfs}
% \usepackage{kpfonts}
% \usepackage[cal=dutchcal,
% bb=boondox,
% bbscaled=1.05,
% scr=boondoxupr]{mathalpha}

\renewcommand{\epsilon}{\varepsilon}
\renewcommand{\phi}{\varphi}

\DeclareMathAlphabet{\mathpzc}{OT1}{pzc}{m}{it}

\newcommand{\goth}[1]{\mathfrak{#1}}
\newcommand{\red}[1]{#1_{\text{red}}}
\newcommand{\reg}[1]{#1_{\text{reg}}}

\newcommand{\fA}{\mathscr{A}}
\newcommand{\fE}{\mathscr{E}}
\newcommand{\fF}{\mathscr{F}}
\newcommand{\fG}{\mathscr{G}}
\newcommand{\fH}{\mathscr{H}}
\newcommand{\fI}{\mathscr{I}}
\newcommand{\fJ}{\mathscr{J}}
\newcommand{\fK}{\mathscr{K}}
\newcommand{\fL}{\mathscr{L}}
\newcommand{\fM}{\mathscr{M}}
\newcommand{\fN}{\mathscr{N}}
\newcommand{\fO}{\mathscr{O}}
\newcommand{\fP}{\mathscr{P}}
\newcommand{\fR}{\mathscr{R}}
\newcommand{\fS}{\mathscr{S}}
\newcommand{\fT}{\mathscr{T}}
\newcommand{\fX}{\mathscr{X}}
\newcommand{\fY}{\mathscr{Y}}
\newcommand{\fZ}{\mathscr{Z}}

\newcommand{\gm}{\goth{m}}
\newcommand{\gF}{\goth{F}}
\newcommand{\gU}{\goth{U}}
\newcommand{\gV}{\goth{V}}

\newcommand{\A}{\mathbf{A}}
\newcommand{\C}{\mathbf{C}}
\newcommand{\F}{\mathbf{F}}
\newcommand{\G}{\mathbf{G}}
\newcommand{\N}{\mathbf{N}}
\newcommand{\PP}{\mathbf{P}}
\newcommand{\Q}{\mathbf{Q}}
\newcommand{\R}{\mathbf{R}}
\newcommand{\Z}{\mathbf{Z}}

\newcommand\srestr[2]{{\left.\kern-\nulldelimiterspace #1\vphantom{\small|} \right|_{#2}}}
\newcommand\restr[2]{{\left.\kern-\nulldelimiterspace #1 \vphantom{\big|} \right|_{#2}}}

\newtheorem{theorem}{Theorem}
\newtheorem{lemma}{Lemma}
\newtheorem{corollary}{Corollary}
\newtheorem{proposition}{Proposition}

\DeclareMathOperator{\rProj}{\mathpzc{Proj}}
\DeclareMathOperator{\rSpec}{\mathpzc{Spec}}
\DeclareMathOperator{\rHom}{\mathpzc{Hom}}
\DeclareMathOperator{\id}{id}
\DeclareMathOperator{\Frac}{Frac}
\DeclareMathOperator{\rk}{rank}
\DeclareMathOperator{\pic}{Pic}
\DeclareMathOperator{\cacl}{CaCl}
\DeclareMathOperator{\trd}{tr.d.}
\DeclareMathOperator{\cl}{Cl}
\DeclareMathOperator{\depth}{depth}
\DeclareMathOperator{\codim}{codim}
\DeclareMathOperator{\Div}{Div}
\DeclareMathOperator{\coker}{coker}
\DeclareMathOperator{\len}{length}
\DeclareMathOperator{\height}{height}
\DeclareMathOperator{\supp}{Supp}
\DeclareMathOperator{\proj}{Proj}
\DeclareMathOperator{\im}{im}
\DeclareMathOperator{\Hom}{Hom}
\DeclareMathOperator{\Der}{Der}
\DeclareMathOperator{\spec}{Spec}
\DeclareMathOperator{\Aut}{Aut}
\DeclareMathOperator{\ch}{char}
\DeclareMathOperator{\tor}{Tor}
\DeclareMathOperator{\Ann}{Ann}
\DeclareMathOperator{\Syl}{Syl}
\DeclareMathOperator{\Sym}{Sym}
\DeclareMathOperator{\GL}{GL}
\DeclareMathOperator{\SL}{SL}
\DeclareMathOperator{\Stab}{Stab}
\DeclareMathOperator{\Perm}{Perm}
\DeclareMathOperator{\Orb}{Orb}
\DeclareMathOperator{\Gal}{Gal}
\DeclareMathOperator{\Supp}{Supp}

\author{James Lee}

% Solarized
% \pagecolor[RGB]{0,20,26}
% \color[RGB]{191,191,191}

% Sephia
\pagecolor[RGB]{249,239,220}

% Black
% \pagecolor[RGB]{20,20,20}
% \color[RGB]{200,200,200}

\title{Chapter 2, Section 6}

\begin{document}
\maketitle
\begin{enumerate} [label=\textbf{\arabic*.}, leftmargin=0em]

\item[\textbf{1.}] Let $X$ be a scheme satisfying ($*$). Then $X \times \PP^n$ also satisfies ($*$), and $\cl (X \times \PP^n) \cong (\cl X) \times \Z$.

\begin{proof}
    Since $\PP^n$ is a union of $n+1$ copies of $\A^n$, it is not hard to see from (6.6) that $X \times \PP^n$ is noetherian and integral. It's left to show it is separated. Indeed, $X \times \PP^n \to X$ is a projective morphism of noetherian schemes, hence it is proper by (4.9). Composition of separated morphisms is separated. Hence, $X \times \PP^n$ satisfies ($*$).

    The projective $n$-space is the union of a hyperplane $\PP^{n - 1}$ and a copy of $\A^n$, and $\PP^{n - 1}$ is a prime divisor of $\PP^n$. Thus, $X \times \PP^{n-1}$ is a prime divisor of $X \times \PP^n$Also, $X \times \PP^n - X \times \PP^{n-1} = X \times \A^n$, and $\cl(X \times \A^n) \cong \cl{X}$. Hence, we have the exact sequence
    \begin{equation*}
        0 \to \Z \xrightarrow{1 \mapsto X \times \PP^{n-1}} \cl(X \times \PP^n) \to \cl{X} \to 0.
    \end{equation*} 
    This exact sequence splits via $Y \in \cl{X} \mapsto Y \times \PP^n \in \cl(X \times \PP^n)$. Injectivity of this map follows from $\PP^n$ being birational to $\A^n$ and (6.6). Hence, $\cl(X \times \PP^n) \cong \cl(X) \times \Z$.
\end{proof}

\item[\textbf{2.}] \textit{Varieties in Projective Space.} Let $k$ be an algebraically closed field, and let $X$ be a closed subvariety of $\PP_k^n$ which is nonsingular in codimension one (hence satisfies ($*$)). For any divisor $D = \sum n_i Y_i$ on $X$, we define the \textit{degree} of $D$ to be $\sum n_i \deg Y_i$, where $\deg Y_i$ is the degree of $Y_i$, considered as a projective variety itself.
\begin{itemize}
    \item[(a)] Let $V$ be an irreducible hypersurface in $\PP^n$ which does not contain $X$, and let $Y_i$ be the irreducible components of $V \cap X$. They all have codimension $1$ by (I, Ex. 1.8). For each $i$, let $f_i$ be a local equation for $V$ on some open set $U_i$ of $\PP^n$ for which $Y_i \cap U_i \neq \emptyset$, and let $n_i = v_{Y_i}(\bar{f}_i)$, where $\bar{f}_i$ is the restriction of $f_i$ to $U_i \cap X$. Then we define the \textit{divisor} $V.X$ to be $\sum n_i Y_i$. Extend by linearity and show that this gives a well-defined homomorphism from the subgroup of $\Div \PP^n$ consisting of divisors, none of whose components contain $X$, to $\Div X$.
    \item[(b)] If $D$ is a principal divisor on $\PP^n$, for which $D.X$ is defined as in (a), show that $D.X$ is principal on $X$. Thus, we get a homomorphism $\cl\PP^n \to \cl X$.
    \item[(c)] Show that the integer $n_i$ defined in (a) is the same as the intersection multiplicity $i(X, V; Y_i)$ defined in (I, \textsection{7}). Then use the generalized Bézout's theorem (I, 7.7) to show that for any divisor $D$ on $\PP^n$, none of whose components contain $X$,
    \begin{equation*}
        \deg(D.X) = (\deg D) \cdot (\deg X).
    \end{equation*}
    \item[(d)] If $D$ is a principal divisor on $X$, show that there is a rational function $f$ on $\PP^n$ such that $D = (f).X$. Conclude that $\deg D = 0$. Thus, the degree function defines a homomorphism $\deg : \cl X \to \Z$. Finally, there is a commutative diagram
    \[\begin{tikzcd}
        \cl \PP^n \arrow[r] \arrow[d, "\cong"'] \arrow[d, "\deg "] & \cl X \arrow[d, "\deg"] \\
        \Z \arrow[r, "\cdot(\deg X)"']                                  & \Z                                 
        \end{tikzcd} \]
    and in particular, we see that the map $\cl \PP^n \to \cl X$ is injective.
\end{itemize}

\begin{proof} $ $ \vspace{0pt}
\begin{itemize} [leftmargin=0cm]
\item[(a)] Let $f_i'$ be another local equation for $V$ on some open set $U_i'$ of $\PP^n$ for which $Y_i \cap U_i' \neq \emptyset$. Since $X$ is irreducible, $U_i \cap U_i' \cap X \neq \emptyset$, and $f_i / f_i' \in \Gamma(U_i \cap U_i', \fO_{\PP^n}^*)$. Hence, $\bar{f}_i / \bar{f}_i' \in \Gamma(U_i \cap U_i' \cap X, \fO_X^*)$. Hence, $n_i$ is independent of the choice of $U_i$.

\item[(b)] Suppose $D = (f)$, and let $i : X \hookrightarrow \PP^n$ be a closed immersion. Then $D.X = (i^{-1}f)$.

\item[(c)] We recall the definition of the intersection multiplicity:  
\[
    i(V, X; Y_i) = \len_{S_{\goth{p}_i}} (S/(I_X + I_V))_{\goth{p}_i},
\]
where \( S = k[x_0, \dots, x_n] \), and \( I_X, I_V \) are the homogeneous ideals defining \( X \) and \( V \), respectively. The homogeneous prime ideal \( \goth{p}_i \) defines \( Y_i \). Note that $\len_{S_{\goth{p}}} M_\goth{p} = \len_{S_{(\goth{p})}} M_{(\goth{p})}$. Let $\goth{P}_i$ be the image of $\goth{p}_i$ in $S / I_X$, and let $A = (S / I_X)_{(\goth{P}_i)}$. Then $A$ is precisely the valuation ring of $Y_i$ as a prime divisor of $X$. If we denote $I'_V$ the extension of the ideal $I_V$ in $S$, it is not hard to see that $\len_A A / I_V' = v_{Y_i}(\overline{f})$, where $f$ is any local equation defining $V$.

Let $D = \sum n_j V_j$ be any $\PP^n$ with each $V_j$ an irreducible hypersurface not containing $X$. By the generalized Bézout's theorem,
\begin{align*}
    \deg(D.X) & = \sum n_j \deg(V_j.X) \\
    & = \sum n_j \sum i(V_j, X ; V_{jk}) \deg{V_{jk}} \\
    & = \sum n_j (\deg{V_j} \cdot \deg{X}) \\
    & = (\deg{D}) \cdot (\deg{X}).
\end{align*}

\item[(d)] Let $D = (f)$ for some $f \in K(X)$, where $K(X)$ is the function field of $X$. If $X$ is locally defined by an ideal $I \subset k[y_1, \dots, y_n]$ on some open subset $\spec{k[y_1, \dots, y_n]}$ of $\PP^n$ where $f$ is regular, then we can write $f = g / h$ for some $g, h \in k[y_1, \dots, y_n]$ such that $h$ is nowhere zero on some open set in $X$, which is what we wanted to show. Since any principal divisor has degree zero, by (c), $\deg(D) = \deg((f).X) = 0 \cdot \deg{X} = 0$.
\end{itemize}
\end{proof}

\item [\textbf{4.}] Let $k$ be a field of characteristic $\neq 2$. Let $f \in k[x_1, \dots, x_n]$ be a \textit{square-free} nonconstant polynomial, i.e., in the unique factorization of $f$ into irreducible polynomials, there are no repeated factors. Let $A = k[x_1, \dots, x_n, z] / (z^2 - f)$. Show that $A$ is an integrally closed ring.

\begin{proof}
    The quotient field $K$ of $A$ is just $k(x_1, \dots, x_n)[\sqrt{f}]$.
    It is a Galois extension of $k[x_1, \dots, x_n]$ with Galois group $\Z / 2 \Z$ generated by $\sqrt{f} \mapsto -\sqrt{f}$.
    If $\alpha = g + h \sqrt{f} \in K$, where $g, h \in k(x_1, \dots, x_n)$, then the minimal polynomial of $\alpha$ is $X^2 - 2g X + (g^2 - h^2f)$. Suppose $\alpha$ is integral over $k[x_1, \dots, x_n]$. Since $k[x_1, \dots, x_n]$ is integrally closed, the coefficients of $p_\alpha$ lie in $k[x_1, \dots, x_n]$ (A.M. 5.15). It follows $g, h \in k[x_1, \dots, x_n]$. Hence, the integral closure of $k[x_1, \dots, x_n]$ in $K$ lies in $A$. The converse is immediate by the formula for the minimal polynomial of any $\alpha \in K$. Hence, $A$ is an integrally closed ring.
\end{proof}

\item[\textbf{7.}] Let $X$ be the nodal cubic curve $y^2 z = x^3 + x^2 z$ in $\PP^2$. Imitate (6.11.4) and show that the group of Cartier divisors of degree $0$, $\cacl^\circ X$, is naturally isomorphic to the multiplicative group $\mathbb{G}_m$.

\begin{proof}
   Take $Z = (0:0:1)$ and $P_0 = (0:1:0)$.
   There is a bijection with the closed points of $X - Z$, the set of non-singular points of $X$, and $\cacl^\circ X$.
   It remains to show $X - Z \cong \mathbb{G}_m$, where $\mathbb{G}_m = \A^1 - \{ 0 \}$ is the multiplicative group defined to be the spectrum of the ring $k[t, t^{-1}]$.
   Looking at the affine subset $z \neq 1$, we see that any line through the singular point $Z$, which has affine coordinates $(0, 0)$, intersects the curve one more time. 
   Plugging in $y = \lambda x$ into the defining equation of $X$ for some $\lambda \in k^*$, we have $x = \lambda^2  - 1$.
   Thus, we have a map $k^* - \{ \pm 1 \} \to X - Z$ defined by $\lambda \mapsto (\lambda^2 - 1, (\lambda^2 - 1)^3 + (\lambda^2 - 1)^2)$.
   Extend this map to all of $k^*$ via $1 \mapsto P_0$ and $-1 \mapsto (-1, 0)$, and we are done.
\end{proof}

\item[\textbf{8.}]
\begin{itemize}
    \item[(a)] Let $f : X \to Y$ be a morphism of schemes. Show that $\fL \mapsto f^* \fL$ induces a homomorphism of Picard groups, $f^* : \pic Y \to \pic X$.

    \item[(b)] If $f$ is a finite morphism of nonsingular curves, show that this homomorphism corresponds to the homomorphism $f^* : \cl Y \to \cl X$ defined in the text, via the isomorphisms of (6.16).

    \item[(c)] If $X$ is a locally factorial integral closed subscheme of $\PP_k^n$, and if $f : X \to \PP^n$ is the inclusion map, then $f^*$ on $\pic$ agrees with the homomorphism on divisor class groups defined in (Ex. 6.2) via the isomorphisms of (6.16).
\end{itemize}

\begin{proof} $ $ \vspace{0pt}
\begin{itemize} [leftmargin=0cm]
\item[(a)] If $V$ is an open set in $Y$ such that $\restr{\fL}{V} \cong \fO_V$, then $\restr{f^*\fL}{f^{-1}V} \cong \fO_{f^{-1}V}$.

\item[(b)] Let $D$ be a Weil divisor on $Y$. We want to show $f^* (\fO_Y(D)) \cong \fO_X(f^*D)$. By the isomorphism $\cacl{X} \cong \cl{X}$, there exists a representation $\{ (U_i, s_i) \}$ of $D$ as a Cartier divisor.
A finite morphism of nonsingular curves induces an inclusion of function fields $K(Y) \to K(X)$.
Since the $s_i$'s are elements of $K(Y)$, by (a), $f^*(\fO_X(D))$ is the Cartier divisor defined by $\{ (f^{-1}U_i, f^{-1}s_i)\}$.
It is clear that the Weil divisor associated to $\{ (f^{-1}U_i, f^{-1}s_i)\}$ is $f^* D$. 

\item[(c)] The map defined in (Ex. 6.2) is precisely $f^* : \cl{Y} \to \cl{X}$ as defined in the text.
\end{itemize} 
\end{proof}

\item[\textbf{10.}] \textit{The Grothendiek Group $K(X)$.} Let $X$ be a noetherian scheme. We define $K(X)$ to be the quotient of the free abelian group generated by all the coherent sheaves on $X$, by the subgroup generated by all expressions $\fF - \fF' - \fF''$, whenever there is an exact sequence $0 \to \fF' \to \fF \to \fF'' \to 0$ of coherent sheaves on $X$. If $\fF$ is a coherent sheaf, we denote by $\gamma(\fF)$ its image in $K(X)$.
\begin{itemize}
    \item[(a)] If $X = \A_k^1$, then $K(X) \cong \Z$.

    \item[(b)] If $X$ is any integral scheme, and $\fF$ a coherent sheaf, we define the \textit{rank} of $\fF$ to be $\dim_K{\fF_\xi}$, where $\xi$ is the generic point of $X$, and $K = \fO_{\xi}$ is the function field of $X$. Show that the rank function defines a surjective homomorphism $\rk : K(X) \to \Z$.

    \item[(c)] If $Y$ is a closed subcheme of $X$, there is an exact sequence
    \[\begin{tikzcd}
        K(Y) \arrow[r] & K(X) \arrow[r] & K(X - Y) \arrow[r] & 0,
        \end{tikzcd} \]
    where the first map is extension by zero, and the second map is restriction.
\end{itemize}

\begin{proof} $ $ \vspace{0pt}
\begin{itemize} [leftmargin=0cm]
\item[(a)] Suppose $X = \spec{A}$ for some principle ideal domain $A$. Coherent sheaves on $X$ correspond to finitely generated $A$-modules. Any such module can be decomposed as a direct sum of a free and torsion submodule. Any torsion module is a direct sum of $A / \goth{p}$, where $\goth{p}$ is a prime ideal of $A$. Since $A$ is a principle ideal domain, all prime ideals are rank one $A$-modules, so $\gamma(A) = \gamma(\goth{p})$. We have the exact sequence
\begin{equation*}
    0 \to \goth{p} \to A \to A / \goth{p} \to 0,
\end{equation*}
which implies $\gamma(A / \goth{p}) = 0$. Hence, we can identify elements of $K(X)$ to the rank of their free parts.

\item[(b)] Take $\fF = \fO_X^{\oplus n}$ for each $n$, then $\rk \fF = n$.

\end{itemize}
\end{proof}

\item[\textbf{11.}] \textit{The Grothendiek Group of a Nonsingular Curve.} Let $X$ be a nonsingular curve over an algebraically closed field $k$. We will show that $K(X) \cong \pic X \oplus \Z$, in several steps.
\begin{itemize}
    \item[(a)] For any divisor $D = \sum n_i P_i$ on $X$, let $\psi(D) = \sum n_i \gamma(k(P_i)) \in K(X)$, where $k(P_i)$ is the skyscraper sheaf $k$ at $P_i$ and $0$ elsewhere. If $D$ is an effective divisor, let $\fO_D$ be the structure sheaf of the associated subscheme of codimension $1$, and show that $\psi(D) = \gamma(\fO_D)$. Then use (6.18) to show that for any $D$, $\psi(D)$ depends only on the linear equivalence class of $D$, so $\psi$ defines a homomorphism $\psi : \cl X \to K(X)$.

    \item[(b)] For any coherent sheaf $\fF$ on $X$, show that there exist locally free sheaves $\fE_0$ and $\fE_1$ and an exact sequence $0 \to \fE_1 \to \fE_0 \to \fF \to 0$. Let $r_0 = \rk \fE_0$, $r_1 = \rk \fE_1$, and define $\det \fF = (\bigwedge^{r_0} \fE_0) \otimes (\bigwedge^{r_1} \fE_1)^{-1} \in \pic X$. Show that $\det \fF$ is independent of the resolution chosen, and that it gives a homomorphism $\det : K(X) \to \pic X$. Finally, show that if $D$ is a divisor, then $\det(\psi(D)) = \fL(D)$.

    \item[(c)] if $\fF$ is any coherent sheaf of rank $r$, show that there is a divisor $D$ on $X$ and an exact sequence $0 \to \fL^{\oplus r}(D) \to \fF \to \mathcal{T} \to 0$, where $\mathcal{T}$ is a torsion sheaf. Conclude that if $\fF$ is a sheaf of rank $r$, then $\gamma(\fF) - r \gamma(\fO_X) \in \im \psi$.

    \item[(d)] Using the maps $\psi, \det, \rk$, and $1 \mapsto \gamma(\fO_X)$ from $\Z \to K(X)$, show that $K(X) \cong \pic X \oplus \Z$. 
\end{itemize}

\begin{proof} $ $ \vspace{0pt}
\begin{itemize}
\item[(a)] Let $D = \sum n_i P_i$ be an effective divisor, and let $\{ (U_i, f_i) \}$ be the effective Cartier divisor associated to $D$, where $f_i \in \Gamma(U_i, \fO_{U_i})$ and $v_{P_i}(f_i) = n_i$ for all $i$. Thus, if $t_i$ is an uniformizing element of $\fO_{P_i, X}$ for each $i$, then $\fO_D$ be can be identified with the direct sum of skyscraper sheaves $\bigoplus \fO_{P_i, X} / (t_i^{n_i})$ on each $P_i$. Note that $ \fO_{P_i, X} / (t_i^{n_i})$ is non-zero for only finitely many $P_i$. Since $k$ is algebraically closed, the residue field of $\fO_{P_i, X}$ is $k$, and since each $\fO_{P_i, X}$ is a discrete valuation ring, $\fO_{P_i, X} / (t_i^{n_i})$ is isomorphic to $k^{\oplus n_i}$ as a $k$-vector space, so $\gamma(\fO_{P_i, X}/(t_i^{n_i})) = \gamma(k(P_i)^{\oplus n_i}) = n_i \gamma(k(P_i)) = \psi(n_iP_i)$, which is what we wanted to show. 

We need to show for any principal divisor $D = (f)$, where $f \in K^*$, the function field of $X$, $\psi(D) = 0$. Any principal divisor can be written as a difference of two effective principal divisors, so we assume $D$ is effective. Let $Y$ be the associated closed subscheme of $D$. Then the ideal sheaf $\fI_Y$ is generated by $f$, so we have an isomorphism of $\fO_X$-modules $\fO_X \to \fI_Y$ defined by multiplication by $f$. Hence, $\psi(D) = \gamma(\fO_D) = \gamma(\fO_X / \fI_Y) = \gamma(\fO_X) - \gamma(\fI_Y) = 0$ by (6.18).

\item[(b)] Let $U_i = \spec{A_i}$ be an open affine cover of $X$ such that $\restr{\fF}{U_i} \cong \widetilde{M}_i$ for some finitely generated $A_i$-module.
Since $X$ is noetherian, a finite number will do, and let $N > 0$ such that the minimal number of generators for $M_i$ is at most $N$ for all $i$.
\end{itemize} 
\end{proof}

\item[\textbf{12.}] Let $X$ be a complete nonsingular curve.
Show that there is a unique way to define the \textit{degree} of any coherent sheaf on $X$, $\deg \fF \in \Z$, such that:
\begin{itemize}
    \item[(1)] If $D$ is a divisor, $\deg \fL(D) = \deg D$.
    \item[(2)] If $\fF$ is a \textit{torsion sheaf} (meaning a sheaf whose stalk at the generic point is zero), then $$\deg \fF = \sum_{P \in X} \len(\fF_P).$$
    \item[(3)] If $0 \to \fF' \to \fF \to \fF'' \to 0$ is an exact sequence, then $\deg \fF = \deg \fF' + \deg \fF''$.
\end{itemize}

\begin{proof} 
    $K(X) \xrightarrow{\det} \pic X \cong \cl{X} \xrightarrow{\deg} \Z$.
\end{proof}

\end{enumerate}

\end{document}

% \item \textit{Cones.} In this exercise we compare the class group of a projective variety $V$ to the class group of its cone. So let $V$ be a projective variety in $\PP^n$ which is of dimension $\geq 1$ and nonsingular in codimension $1$. Let $X = C(V)$ be the affine cone over $V$ in $\mathbb{A}^{n + 1}$, and let $\bar{X}$ be its projective closure in $\PP^{n + 1}$. Let $P \in X$ be the vertex of the cone.
% \begin{itemize}
%     \item[(a)] Let $\pi : \bar{X} - P \to V$ be the projection map. Show that $V$ can be covered by open subsets $U_i$ such that $\pi^{-1}(U_i) \cong U_i \times \mathbb{A}^1$ for each $i$, and then show as in (6.6) that $\pi^* : \cl V \to \cl(\bar{X} - P)$ is an isomorphism. Since $\cl~\bar{X} \simeq \cl(\bar{X} - P)$, we have also $\cl V \simeq \cl~\bar{X}$.
%     \item[(b)] We have $V \subseteq \bar{X}$ as the hyperplane section at infinity. Show that the class of the divisor $V$ in $\cl \bar{X}$ is equal to $\pi^*$ (class of $V.H$) where $H$ is any hyperplane of $\PP^n$ not containing $V$. Thus conclude using (6.5) that there is an exact sequence
%     \begin{equation*}
%         0 \to \Z \to \cl V \to \cl X \to 0,
%     \end{equation*}
%     where the first arrow sends $1 \mapsto V.H$, and the second is $\pi^*$ followed by the restriction to $X - P$ and inclusion in $X$.
%     \item[(c)] Let $S(V)$ be the homogenous coordinate ring of $V$ (which is also the affine coordinate ring of $X$). Show that $S(V)$ is a unique factorization domain if and only if
%     \begin{itemize}
%         \item[(1)] $V$ is projectively normal, and
%         \item[(2)] $\cl V \simeq \Z$ and is generated by the class of $V.H$.
%     \end{itemize}
% \end{itemize}

% \item \textit{Quadric Hypersurfaces.} Let $\text{char}~k \neq 2$, and let $X$ be the affine quadric hypersurface defined by $x_0^2 + x_1^2 + \cdots + x_r^2 = 0$.
% \begin{itemize}
%     \item[(a)] Show that $X$ is normal if $r \geq 2$.
%     \item[(b)] Show by a suitable linear change of coordinates that the equation of $X$ could be written as $x_0 x_1 = x_2^2 + \cdots + x_r^2$. Now imitate the method of (6.5.2) to show that:
%     \begin{itemize}
%         \item[(1)] If $r = 2$, then $\cl X \simeq \Z / 2 \Z$;
%         \item[(2)] If $r = 3$, then $\cl X \simeq \Z$;
%         \item[(3)] If $r \geq 4$, then $\cl X = 0$.
%     \end{itemize}
%     \item[(c)] Now let $Q$ be the projective quadric hypersurface in $\PP^n$ defined by the same equation. Show that
%     \begin{itemize}
%         \item[(1)] If $r = 2$, $\cl Q \simeq \Z$, and the class of a hyperplane section $Q.H$ is twice the generator;
%         \item[(2)] If $r = 3$, $\cl Q \simeq \Z \oplus \Z$;
%         \item[(3)] If $r \geq 4$, $\cl Q \simeq \Z$, generated by $Q.H$.
%     \end{itemize}
%     \item[(d)] Prove Klein's theorem, which says that if $r \geq 4$, and if $Y$ is an irreducible subvariety of codimension $1$ on $Q$, then there is an irreducible hypersurface $V \subseteq \PP^n$ such that $V \cap Q = Y$, with multiplicity one. In other words, $Y$ is a complete intersection.
% \end{itemize}

% \item Let $X$ be the nonsingular plane cubic curve $y^2 z = x^3 - xz^2$ of (6.10.2).
% \begin{itemize}
%     \item[(a)] Show that three points $P, Q, R$ of $X$ are collinear if and only if $P + Q + R = 0$ in the group law on $X$. (Note that the point $P_0 = (0, 1, 0)$ is the zero element in the group structure on $X$.)
%     \item[(b)] A point $P \in X$ has order $2$ in the group law on $X$ if and only if the tangent line at $P$ passes through $P_0$.
%     \item[(c)] A point $P \in X$ has order $3$ in the group law on $X$ if and only if $P$ is an inflection point. (An \textit{inflection point} of a plane curve is a nonsingular point $P$ of the curve, whose tangent line (\S 1, Ex. 7.3) has intersection multiplicity $\geq 3$ with the curve at $P$.)
%     \item[(d)] Let $k = \C$. Show that the points of $X$ with coordinates in $\Q$ form a subgroup of the group $X$. Can you determine the structure of this subgroup explicitly?
% \end{itemize}

% \item \textit{Singular Curves.} Here we give another method of calculating the Picard group of a singular curve. Let $X$ be a projective curve over $k$, let $\widetilde{X}$ be its normalization, and let $\pi : \widetilde{X} \to X$ be the projection map. For each point $P \in X$, let $\fO_P$ be its local ring, and let $\widetilde{\fO}_P$ be the integral closure of $\fO_P$. We use a * to denote the group of units in a ring.
% \begin{itemize}
%     \item[(a)] Show that there is an exact sequence
%     \begin{equation*}
%         0 \to \bigoplus_{P \in X} \widetilde{\fO}_P^* / \fO_P^* \to \pic X \xrightarrow{\pi^*} \pic \widetilde{X} \to 0.
%     \end{equation*}
%     \item[(b)] Use (a) to give another proof of the fact that if $X$ is a plane cuspidal cubic curve, then there is an exact sequence
%     \begin{equation*}
%         0 \to \mathbb{G}_a \to \pic X \to \Z \to 0,
%     \end{equation*}
%     and if $X$ is a plane nodal cubic curve, there is an exact sequence
%     \begin{equation*}
%         0 \to \mathbb{G}_m \to \pic X \to \Z \to 0.
%     \end{equation*}
% \end{itemize}
