\documentclass{article}
\usepackage[margin=1in]{geometry}
\usepackage{amsmath,amsfonts,mathtools,amsthm,amssymb}
\usepackage[l2tabu,orthodox]{nag}
\usepackage{microtype} % fixes spacing or whatever
\usepackage{enumitem}
\usepackage{tikz-cd}
\usepackage{xcolor}
\usepackage{physics}
\usepackage{mathrsfs}
% \usepackage{kpfonts}
% \usepackage[cal=dutchcal,
% bb=boondox,
% bbscaled=1.05,
% scr=boondoxupr]{mathalpha}

\renewcommand{\epsilon}{\varepsilon}
\renewcommand{\phi}{\varphi}

\DeclareMathAlphabet{\mathpzc}{OT1}{pzc}{m}{it}

\newcommand{\goth}[1]{\mathfrak{#1}}
\newcommand{\red}[1]{#1_{\text{red}}}
\newcommand{\reg}[1]{#1_{\text{reg}}}

\newcommand{\fA}{\mathscr{A}}
\newcommand{\fE}{\mathscr{E}}
\newcommand{\fF}{\mathscr{F}}
\newcommand{\fG}{\mathscr{G}}
\newcommand{\fH}{\mathscr{H}}
\newcommand{\fI}{\mathscr{I}}
\newcommand{\fJ}{\mathscr{J}}
\newcommand{\fK}{\mathscr{K}}
\newcommand{\fL}{\mathscr{L}}
\newcommand{\fM}{\mathscr{M}}
\newcommand{\fN}{\mathscr{N}}
\newcommand{\fO}{\mathscr{O}}
\newcommand{\fP}{\mathscr{P}}
\newcommand{\fR}{\mathscr{R}}
\newcommand{\fS}{\mathscr{S}}
\newcommand{\fT}{\mathscr{T}}
\newcommand{\fX}{\mathscr{X}}
\newcommand{\fY}{\mathscr{Y}}
\newcommand{\fZ}{\mathscr{Z}}

\newcommand{\gm}{\goth{m}}
\newcommand{\gF}{\goth{F}}
\newcommand{\gU}{\goth{U}}
\newcommand{\gV}{\goth{V}}

\newcommand{\A}{\mathbf{A}}
\newcommand{\C}{\mathbf{C}}
\newcommand{\F}{\mathbf{F}}
\newcommand{\G}{\mathbf{G}}
\newcommand{\N}{\mathbf{N}}
\newcommand{\PP}{\mathbf{P}}
\newcommand{\Q}{\mathbf{Q}}
\newcommand{\R}{\mathbf{R}}
\newcommand{\Z}{\mathbf{Z}}

\newcommand\srestr[2]{{\left.\kern-\nulldelimiterspace #1\vphantom{\small|} \right|_{#2}}}
\newcommand\restr[2]{{\left.\kern-\nulldelimiterspace #1 \vphantom{\big|} \right|_{#2}}}

\newtheorem{theorem}{Theorem}
\newtheorem{lemma}{Lemma}
\newtheorem{corollary}{Corollary}
\newtheorem{proposition}{Proposition}

\DeclareMathOperator{\rProj}{\mathpzc{Proj}}
\DeclareMathOperator{\rSpec}{\mathpzc{Spec}}
\DeclareMathOperator{\rHom}{\mathpzc{Hom}}
\DeclareMathOperator{\id}{id}
\DeclareMathOperator{\Frac}{Frac}
\DeclareMathOperator{\rk}{rank}
\DeclareMathOperator{\pic}{Pic}
\DeclareMathOperator{\cacl}{CaCl}
\DeclareMathOperator{\trd}{tr.d.}
\DeclareMathOperator{\cl}{Cl}
\DeclareMathOperator{\depth}{depth}
\DeclareMathOperator{\codim}{codim}
\DeclareMathOperator{\Div}{Div}
\DeclareMathOperator{\coker}{coker}
\DeclareMathOperator{\len}{length}
\DeclareMathOperator{\height}{height}
\DeclareMathOperator{\supp}{Supp}
\DeclareMathOperator{\proj}{Proj}
\DeclareMathOperator{\im}{im}
\DeclareMathOperator{\Hom}{Hom}
\DeclareMathOperator{\Der}{Der}
\DeclareMathOperator{\spec}{Spec}
\DeclareMathOperator{\Aut}{Aut}
\DeclareMathOperator{\ch}{char}
\DeclareMathOperator{\tor}{Tor}
\DeclareMathOperator{\Ann}{Ann}
\DeclareMathOperator{\Syl}{Syl}
\DeclareMathOperator{\Sym}{Sym}
\DeclareMathOperator{\GL}{GL}
\DeclareMathOperator{\SL}{SL}
\DeclareMathOperator{\Stab}{Stab}
\DeclareMathOperator{\Perm}{Perm}
\DeclareMathOperator{\Orb}{Orb}
\DeclareMathOperator{\Gal}{Gal}
\DeclareMathOperator{\Supp}{Supp}

\author{James Lee}

% Solarized
% \pagecolor[RGB]{0,20,26}
% \color[RGB]{191,191,191}

% Sephia
\pagecolor[RGB]{249,239,220}

% Black
% \pagecolor[RGB]{20,20,20}
% \color[RGB]{200,200,200}

\title{Chapter 2, Section 7}

\begin{document}
\maketitle
\begin{enumerate} [label=\textbf{\arabic*.}, leftmargin=-0em]

\item[\textbf{1.}] Let $(X, \fO_X)$ be a locally ringed space, and let $f : \fL \to \fM$ be a surjective map of invertible sheaves on $X$. Show that $f$ is an isomorphism.

\begin{proof}
    We reduce to the following algebraic problem: Any $A$-linear surjective map $\phi : A \to A$ is an isomorphism.
    Indeed, $\phi$ is determined by $r = \phi(1)$, and $\phi$ surjective implies there exists $a \in A$ such that $ar = \phi(a) = 1$. Thus, $r$ is a unit, and $\phi$ is injective.
\end{proof}

\item[\textbf{2.}] Let $X$ be a scheme over a field $k$. Let $\fL$ be an invertible sheaf on $X$, and let $\{ s_0, \dots, s_n \}$ and $\{ t_0, \dots, t_m \}$ be two sets of sections of $\fL$, which generate the same subspace $V \subseteq \Gamma(X, \fL)$, and which generate the sheaf $\fL$ at every point. Suppose $n \leq m$. Show that the corresponding morphisms $\varphi : X \to \PP_k^n$ and $\psi : X \to \PP_k^m$ differ by a suitable linear projection $\PP^m - L \to \PP^n$ and an automorphism of $\PP^n$, where $L$ is a linear subspace of $\PP^m$ of dimension $m - n - 1$.

\item[\textbf{3.}] Let $\varphi : \PP_k^n \to \PP_k^m$ be a morphism. Then:
\begin{itemize}
    \item[(a)] either $\varphi(\PP^n) = pt$ or $m \geq n$ and $\dim{\varphi(\PP^n)} = n$;
    \item[(b)] in the second case, $\varphi$ can be obtained as the composition of (1) a $d$-uple embedding $\PP^n \to \PP^N$ for a uniquely determined $d \geq 1$, (2) a linear projection $\PP^N - L \to \PP^m$, and an automorphism of $\PP^m$. Also, $\varphi$ has finite fibers.
\end{itemize}

\begin{proof} $ $ \vspace{0pt}
\begin{itemize} [leftmargin=0cm]
\item[(a)] Let $\fL$ be the invertible sheaf on $\PP^n$ associated to the morphism $\varphi : \PP^n \to \PP^m$, which is generated by global sections $s_i = \varphi^*(x_i)$, $i = 0, 1, \dots, m$.
Pick $P \in \varphi^{-1}((1, 0, \dots, 0))$ (we can assume such $P$ always exists by applying an automorphism of $\PP^m$).
Identifying $\fL_P \cong \fO_{P, \PP^n}$, we see that $(s_i)_P$ for $i \neq 0$ is not a unit, and they generate the local ring of $\fO_{P, \PP^n}$.
Since $\PP^n$ is smooth, the minimal number of generators of the maximal ideal of $\fO_{P, \PP^n}$ is equal to the dimension $n$.
The sections $s_i$ generate $\fO_{P, \PP^n}$.
Hence, $m \geq n$.
Our proof also shows the induced homomorphism of local rings $\varphi_P : \fO_{\varphi(P), \PP^m} \to \fO_{P, \PP^n}$ is surjective, which implies $\fO_{\varphi(P), \varphi(\PP^n)} = \fO_{\varphi(P), \PP^m} / \ker{\varphi_P} \cong \fO_{P, \PP^n}$.
Hence, $\varphi(P)$ has dimension $n$.
\end{itemize}
\end{proof}

\item[\textbf{5.}] Establish the following properties of ample and very ample invertible sheaves on a noetherian scheme $X$. $\fL, \fM$ will denote invertible sheaves, and for (d), (e) we assume furthermore that $X$ is of finite type over a noetherian ring $A$.
\begin{itemize}
    \item[(a)] If $\fL$ is ample and $\fM$ is generated by global sections, then $\fL \otimes M$ is ample.
    \item[(b)] If $\fL$ is ample and $\fM$ is arbitrary, then $\fM \otimes \fL^n$ is ample for sufficiently large $n$.
    \item[(c)] If $\fL, \fM$ are both ample, so is $\fL \otimes \fM$.
    \item[(d)] If $\fL$ is very ample and $\fM$ is generated by global sections, then $\fL \otimes \fM$ is very ample.
    \item[(e)] If $\fL$ is ample, then there is an $n_0 > 0$ such that $\fL^n$ is very ample for all $n \geq n_0$.
\end{itemize}

\begin{proof} $ $ \vspace{0pt}
\begin{itemize} [leftmargin=0cm]
\item[(a)] Fix a coherent sheaf $\fF$ on $X$ for the remainder of this problem. Let $n \gg 0$ such that $\fF \otimes \fL^n$ is generated by global sections.
If $\fM$ is generated by global sections, then so is $\fM^n$ for any $n > 0$.
More generally, the tensor product of two invertible sheaves generated by global sections is generated by global sections.
Indeed, if $t_i$ and $r_j$ are generating global sections of $\fL$ and $\fM$, then $\fL \otimes \fM$ is generated by $t_i \otimes r_j$ for all pairs $(i, j)$ (tensor products commutes with colimits). 
hus, $\fF \otimes \fL^n \otimes \fM^n \cong \fF \otimes (\fL \otimes \fM)^n$ is generated by global sections. Hence, $\fL \otimes \fM$ is ample.

\item[(b)] Let $n > 0$ such that $\fM \otimes \fL^{n - 1}$ is generated by global sections. By (a), $\fL \otimes (\fM \otimes \fL^{n - 1}) = \fM \otimes \fL^n$ is ample.

\item[(c)] If $\fM$ is ample, then $\fM^n$ is generated by global sections for large enough $n$. Also, $\fL^n$ is ample by (II, 7.5), so $\fL^n \otimes \fM^n$ is ample. Hence, $\fL \otimes \fM$ is ample by (II, 7.5) again.

\item[(d)] Let $i : X \to \PP_A^n$ be an immersion such that $\fL \cong i^*\fO_{\PP_A^n}(1)$, and let $\varphi : X \to \PP_A^m$ be the unique $A$-morphism corresponding to $\fM$ (II, 7.1).
By the universal property of the fiber project $\PP_A^n \times_A \PP_A^m$, there exists a unique $A$-morphism $\phi : X \to \PP_A^n \times_A \PP_A^m$ such that $i = p_n \circ \phi$ and $\varphi = p_m \circ \phi$, where $p_n : \PP_A^n \times_A \PP_A^m \to \PP_A^n$ and $p_m : \PP_A^n \times_A \PP_A^m \to \PP_A^m$ are the natural projection maps.
It is not hard to see $\phi$ is an immersion, and composing it with a Segre embedding (Ex. 5.12), we obtain an immersion $\phi' : X \to \PP_A^N$, where $N = nm + n + m$, such that $\fL \otimes \fM \cong \phi'^* \fO(1)$.

\item[(e)] If $\fL$ is ample, then there exists $n_0 > 0$ such that $\fL^{n - 1}$ is generated by global sections for all $n \geq n_0$. By (d), $\fL^n = \fL \otimes \fL^{n - 1}$ is very ample for all $n \geq n_0$.
\end{itemize} 
\end{proof}

\end{enumerate}
\end{document}

% \item \begin{itemize}
%     \item[(a)] Use (7.6) to show that if $X$ is a scheme of finite type over a noetherian ring $A$, and if $X$ admits an ample invertible sheaf, then $X$ is separated.
%     \item[(b)] Let $X$ be the affine line over a field $k$ with the origin doubled (4.0.1). Calculate $\pic{X}$, determine which invertible sheaves are generated by global sections, and then show directly (without using (a)) that there is no ample invertible sheaf on $X$.
% \end{itemize}

% \item \textit{The Riemann-Roch Problem.} Let $X$ be a nonsingular projective variety over an algebraically closed field, and let $D$ be a divisor on $X$. For any $n > 0$ we consider the complete linear system $|nD|$. Then the Riemann-Roch problem is to determine $\dim{|nD|}$ as a function of $n$, and, in particular, its behavior for large $n$. If an equivalent problem is to determine $\dim{\Gamma(X, \fL^n)}$ as a function of $n$.
% \begin{itemize}
%     \item[(a)] Show that if $D$ is very ample, and if $X \hookrightarrow \PP^n_k$ is the corresponding embedding in projective space, then for all $n$ sufficiently large, $\dim{|nD|} = P_X(n) - 1$, where $P_X$ is the \textit{Hilbert polynomial} of $X$ (I, \S 7). Thus, in this case $\dim{|nD|}$ is a polynomial function of $n$, for $n$ large.
%     \item[(b)] If $D$ corresponds to a torsion element of $\pic{X}$, or order $r$, then $\dim{|nD|} = 0$ if $r \div n$, -1 otherwise. In this case the function is periodic of period $r$.
% \end{itemize}

% It follows from the general Riemann-Roch theorem that $\dim{|nD|}$ is a polynomial function for $n$ large, whenever $D$ is an \textit{ample} divisor. See (IV, 1.3.2), (V, 1.6), and Appendix $A$. In the case of algebraic surfaces, Zariski [7] has shown for any effective divisor $D$, that there is a finite set of polynomials $P_1, \dots, P_r$, such that for all $n$ sufficiently large, $\dim{|nD|} = P_{i(n)}(n)$, where $i(n) \in \{1, 2, \dots, r \}$ is a function of $n$.

% \item \textit{Some Rational Surfaces.} Let $X = \PP_k^2$, and let $|D|$ be the complete linear system of all divisors of degree $2$ on $X$ (conics). $D$ corresponds to the invertible sheaf $\fO(2)$, whose space of global sections has a basis $x^2, y^2, z^2, xy, xz, yz$, where $x, y, z$ are the homogenous coordinates on $X$.
% \begin{itemize}
%     \item[(a)] The complete linear system $|D|$ gives an embedding of $\PP^2$ in $\PP^5$, whose image is the Veronese surface (I, Ex. 2.13).
%     \item[(b)] Show that the subsystem defined by $x^2, y^2, z^2, y(x - z), (x - y)z$ gives a closed immersion of $X$ into $\PP^4$. The image is called the \textit{Veronese surface} in $\PP^4$.
%     \item[(c)] Let $\goth{d} \subseteq |D|$ be the linear system of all conics passing through a fixed point $P$. Then $\goth{d}$ gives an immersion of $U = X - P$ into $\PP^4$. Furthermore, if we blow up $P$, to get a surface $\tilde{X}$, then this map extends to give a closed immersion of $\widetilde{X}$ in $\PP^4$. Show that $\tilde{X}$ is a surface of degree $3$ in $\PP^4$, and that the lines in $X$ through $P$ are transformed into straight lines in $\widetilde{X}$ which do not meet. $\tilde{X}$ is the union of all these liens, so we say $\widetilde{X}$ is a \textit{ruled surface} (V, 2.19.1).
% \end{itemize}

% \item Let $X$ be a noetherian scheme, let $\fE$ be a coherent locally free sheaf on $X$, and let $\pi : \PP(\fE) \to X$ be the corresponding projective space bundle. Show that there is a natural one-to-one correspondence between \textit{sections} of $\pi$ (i.e., morphisms $\sigma : X \to \PP(\fE)$ such that $\pi \circ \sigma = \id_X$) and quotient invertible sheaves $\fE \to \fL \to 0$ of $\fE$.

% \item Let $X$ be a regular noetherian scheme, and $\fE$ a locally free coherent sheaf of rank $\geq 2$ on $X$.
% \begin{itemize}
%     \item[(a)] Show that $\pic{\PP(\fE)} \cong \pic{X} \times \Z$.
%     \item[(b)] If $\fE'$ is another locally free coherent sheaf on $X$, show that $\PP(\fE) \cong \PP(\fE')$ (over $X$) if and only if there is an invertible sheaf $\fL$ on $X$ such that $\fE' \cong \fE \otimes \fL$.
% \end{itemize}

% \item \textit{$\PP^n$-Bundles Over a Scheme.} Let $X$ be a noetherian scheme.
% \begin{itemize}
%     \item[(a)] By analogy with the definition of a vector bundle (Ex. 5.18), define the notion of a \textit{projective $n$-space bundle} over $X$, as a scheme $P$ with a morphism $\pi : P \to X$ such that $P$ is locally isomorphic to $U \times \PP^n$, $U \subseteq X$ open, and the transition automorphisms on $\spec{A} \times \PP^n$ are given by $A$-linear automorphisms of the homogenous coordinate ring $A[x_0, \dots, x_n]$ (e.g., $x_i' = \sum a_{ij} x_j, a_{ij} \in A$).
%     \item[(b)] If $\fE$ is a locally free sheaf of rank $n + 1$ on $X$, then $\PP(\fE)$ is a $\PP^n$-bundle over $X$.
%     \item[(c)] Assume that $X$ is regular, and show that every $\PP^n$-bundle $P$ over $X$ is isomorphic to $\PP(\fE)$ for some locally free sheaf $\fE$ on $X$. Can you weaken the hypothesis ``$X$ regular''?
%     \item[(d)] Conclude (in the case $X$ regular) that we have a one-to-one correspondence between $\PP^n$-bundles over $X$, and equivalence classes of locally free sheaves $\fE$ of rank $n + 1$ under the equivalence relation $\fE' \sim \fE$ if and only if $\fE' \cong \fE \otimes \fM$ for some invertible sheaf $\fM$ on $X$.
% \end{itemize}

% \item On a noetherian scheme $X$, different sheaves of ideals can give rise to isomorphic blow up schemes.
% \begin{itemize}
%     \item[(a)] If $\fI$ is any coherent sheaf of ideals on $X$, show that blowing up $\fI^d$ for any $d \geq 1$ gives a scheme isomorphic to the blowing up of $\fI$ (cf. Ex. 5.13).
%     \item[(b)] If $\fI$ is any coherent sheaf of ideals, and if $\fJ$ is an invertible sheaf of ideals, then $\fI$ and $\fI \cdot \fJ$ give isomorphic blowings-up.
%     \item[(c)] If $X$ is regular, show that (7.17) can be strengthened as follows. Let $U \subseteq X$ be the largest open set such that $f : f^{-1}U \to U$ is an isomorphism. Then $\fI$ can be chosen such that the corresponding closed subscheme $Y$ has support equal to $X - U$.
% \end{itemize}

% \item Let $X$ be a noetherian scheme, and let $Y, Z$ be two closed subschemes, neither one containing the other. Let $\tilde{X}$ be obtained by the blowing up $Y \cap Z$ (defined by the ideal sheaf $\fI_Y + \fI_Z$). Show that the strict transforms $\tilde{Y}$ and $\tilde{Z}$ of $Y$ and $Z$ in $\tilde{X}$ do not meet.

% \item \textit{A Complete Nonprojective Variety.} Let $k$ be an algebraically closed field of $\text{char} \neq 2$. Let $C \subseteq \PP_k^2$ be the nodal cubic curve $y^2 z = x^3 + x^2 z$. If $P_0 = (0, 0, 1)$ is the singular point, then $C - P_0$ is isomorphic to the multiplicative group $\mathbf{G}_m = \spec{k[t, t^{-1}]}$. For each $a \in k$, $a \neq 0$, consider the translation of $\mathbf{G}_m$ given by $t \mapsto at$. This induces an automorphism of $C$ which we denote by $\varphi_a$.

% Now consider $C \times (\PP^1 - \{ 0 \})$ and $C \times (\PP^1 - \{\infty\})$. We glue their open subsets $C \times (\PP^1-\{0, \infty\})$ by the isomorphism $\varphi : \langle P, u \rangle \mapsto \langle \varphi_u(P), u \rangle$ for $P \in C$, $u \in \mathbb{G}_m = \PP^1 - \{0 , \infty \}$. Thus we obtain a scheme $X$, which is our example. The projections to the second factor are compatible with $\varphi$, so there is a natural morphism $\pi : X \to \PP^1$.
% \begin{itemize}
%     \item[(a)] Show that $\pi$ is a proper morphism, and hence that $X$ is a complete variety over $k$.
%     \item[(b)] Use the method of (Ex. 6.9) to show that $\pic{C \times \A^1} \cong \mathbb{G}_m \times \Z$ and $\pic(C \times (\A^1 - \{0\})) \cong \mathbb{G}_m \times \Z \times \Z$.
%     \item[(c)] Now show that the restriction map $\pic{C \times \A^1} \to \pic{C \times (\A^1 - \{0 \})}$ is of the form $\langle t, n \rangle \mapsto \langle t, 0, n \rangle$, and that the automorphism $\varphi$ of $C \times (\A^1 - \{ 0 \})$ induces a map of the form $\langle t, d, n \rangle \mapsto \langle t, d + n, n \rangle$ on its Picard group.
%     \item[(d)] Conclude that the image of the restriction map $\pic{X} \to \text{Pic}(C \times \{0 \})$ consists entirely of divisors of degree $0$ on $C$. Hence $X$ is not projective over $k$ and $\pi$ is not a projective morphism.
% \end{itemize}

% \item \begin{itemize}
%     \item[(a)] Given an example of a noetherian scheme $X$ and a locally free coherent sheaf $\fE$ such that the invertible sheaf $\fO(1)$ on $\PP(\fE)$ is \textit{not} very ample relative to $X$.
%     \item[(b)] Let $f : X \to Y$ be a morphism of finite type, let $\fL$ be an ample invertible sheaf on $X$, and let $\fT$ be a sheaf of graded $\fO_X$-algebras satisfying (\dag). Let $P = \bProj{\fT}$, let $\pi : P \to X$ be the projection, and let $\fO_P(1)$ be the associated invertible sheaf. Show that for all $n \gg 0$, the sheaf $\fO_P(1) \otimes \pi^* \fL^n$ is very ample on $P$ relative to $Y$.
% \end{itemize}
