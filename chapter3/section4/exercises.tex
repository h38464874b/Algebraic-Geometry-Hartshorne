\documentclass{article}
\usepackage[margin=1in]{geometry}
\usepackage{amsmath,amsfonts,mathtools,amsthm,amssymb}
\usepackage[l2tabu,orthodox]{nag}
\usepackage{microtype} % fixes spacing or whatever
\usepackage{enumitem}
\usepackage{tikz-cd}
\usepackage{xcolor}
\usepackage{physics}
\usepackage{mathrsfs}
% \usepackage{kpfonts}
% \usepackage[cal=dutchcal,
% bb=boondox,
% bbscaled=1.05,
% scr=boondoxupr]{mathalpha}

\renewcommand{\epsilon}{\varepsilon}
\renewcommand{\phi}{\varphi}

\DeclareMathAlphabet{\mathpzc}{OT1}{pzc}{m}{it}

\newcommand{\goth}[1]{\mathfrak{#1}}
\newcommand{\red}[1]{#1_{\text{red}}}
\newcommand{\reg}[1]{#1_{\text{reg}}}

\newcommand{\fA}{\mathscr{A}}
\newcommand{\fE}{\mathscr{E}}
\newcommand{\fF}{\mathscr{F}}
\newcommand{\fG}{\mathscr{G}}
\newcommand{\fH}{\mathscr{H}}
\newcommand{\fI}{\mathscr{I}}
\newcommand{\fJ}{\mathscr{J}}
\newcommand{\fK}{\mathscr{K}}
\newcommand{\fL}{\mathscr{L}}
\newcommand{\fM}{\mathscr{M}}
\newcommand{\fN}{\mathscr{N}}
\newcommand{\fO}{\mathscr{O}}
\newcommand{\fP}{\mathscr{P}}
\newcommand{\fR}{\mathscr{R}}
\newcommand{\fS}{\mathscr{S}}
\newcommand{\fT}{\mathscr{T}}
\newcommand{\fX}{\mathscr{X}}
\newcommand{\fY}{\mathscr{Y}}
\newcommand{\fZ}{\mathscr{Z}}

\newcommand{\gm}{\goth{m}}
\newcommand{\gF}{\goth{F}}
\newcommand{\gU}{\goth{U}}
\newcommand{\gV}{\goth{V}}

\newcommand{\A}{\mathbf{A}}
\newcommand{\C}{\mathbf{C}}
\newcommand{\F}{\mathbf{F}}
\newcommand{\G}{\mathbf{G}}
\newcommand{\N}{\mathbf{N}}
\newcommand{\PP}{\mathbf{P}}
\newcommand{\Q}{\mathbf{Q}}
\newcommand{\R}{\mathbf{R}}
\newcommand{\Z}{\mathbf{Z}}

\newcommand\srestr[2]{{\left.\kern-\nulldelimiterspace #1\vphantom{\small|} \right|_{#2}}}
\newcommand\restr[2]{{\left.\kern-\nulldelimiterspace #1 \vphantom{\big|} \right|_{#2}}}

\newtheorem{theorem}{Theorem}
\newtheorem{lemma}{Lemma}
\newtheorem{corollary}{Corollary}
\newtheorem{proposition}{Proposition}

\DeclareMathOperator{\rProj}{\mathpzc{Proj}}
\DeclareMathOperator{\rSpec}{\mathpzc{Spec}}
\DeclareMathOperator{\rHom}{\mathpzc{Hom}}
\DeclareMathOperator{\id}{id}
\DeclareMathOperator{\Frac}{Frac}
\DeclareMathOperator{\rk}{rank}
\DeclareMathOperator{\pic}{Pic}
\DeclareMathOperator{\cacl}{CaCl}
\DeclareMathOperator{\trd}{tr.d.}
\DeclareMathOperator{\cl}{Cl}
\DeclareMathOperator{\depth}{depth}
\DeclareMathOperator{\codim}{codim}
\DeclareMathOperator{\Div}{Div}
\DeclareMathOperator{\coker}{coker}
\DeclareMathOperator{\len}{length}
\DeclareMathOperator{\height}{height}
\DeclareMathOperator{\supp}{Supp}
\DeclareMathOperator{\proj}{Proj}
\DeclareMathOperator{\im}{im}
\DeclareMathOperator{\Hom}{Hom}
\DeclareMathOperator{\Der}{Der}
\DeclareMathOperator{\spec}{Spec}
\DeclareMathOperator{\Aut}{Aut}
\DeclareMathOperator{\ch}{char}
\DeclareMathOperator{\tor}{Tor}
\DeclareMathOperator{\Ann}{Ann}
\DeclareMathOperator{\Syl}{Syl}
\DeclareMathOperator{\Sym}{Sym}
\DeclareMathOperator{\GL}{GL}
\DeclareMathOperator{\SL}{SL}
\DeclareMathOperator{\Stab}{Stab}
\DeclareMathOperator{\Perm}{Perm}
\DeclareMathOperator{\Orb}{Orb}
\DeclareMathOperator{\Gal}{Gal}
\DeclareMathOperator{\Supp}{Supp}

\author{James Lee}

% Solarized
% \pagecolor[RGB]{0,20,26}
% \color[RGB]{191,191,191}

% Sephia
\pagecolor[RGB]{249,239,220}

% Black
% \pagecolor[RGB]{20,20,20}
% \color[RGB]{200,200,200}

\title{Chapter 3, Section 4}

\begin{document}
\maketitle
\begin{enumerate} [label=\textbf{\arabic*.}, leftmargin=0em]

\item Let $f : X \to Y$ be an affine morphism of Noetherian separated schemes (II, Ex. 5.17).
Show that for any quasi-coherent sheaf $\fF$ on $X$, there are natural isomorphisms for all $i \geq 0$,
\begin{equation*}
  H^i(X, \fF) \cong H^i(Y, f_* \fF).
\end{equation*}

\begin{proof}
  Ahhh!
\end{proof}

\item Prove Chevalley's theorem: Let $f : X \to Y$ be a finite surjective morphism of Noetherian separated schemes, with $X$ affine.
Then $Y$ is affine.
\begin{itemize}
  \item[(a)] Let $f : X \to Y$ be a finite surjective morphism of integral Noetherian schemes.
  Show that there is a coherent sheaf $\fM$ on $X$, and a morphism of sheaves $\alpha : \fO_Y^r \to f_* \fM$ for some $r > 0$, such that $\alpha$ is an isomorphism at the generic point of $Y$.

  \item[(b)] For any coherent sheaf $\fF$ on $Y$, show that there is a coherent sheaf $\fG$ on $X$, and a morphism $\beta : f_* \fG \to \fF^r$ which is an isomorphism at the generic point of $Y$. 

  \item[(c)] Now prove Chevalley's theorem.
\end{itemize}

\begin{proof} $ $ \vspace{0pt}
\begin{enumerate} [label=(\alph*), leftmargin=0cm]
\item 
\end{enumerate} 
\end{proof}

\item Let $X = \A_k^2 = \spec{k[x, y]}$, and let $U = X - \{ (0, 0) \}$.
Using a suitable cover of $U$ by open affine subsets, show that $H^1(U, \fO_U)$ is isomorphic to the $k$-vector space spanned by $\{ x^i y^j \mid i, j < 0 \}$.
In particular, it is infinite-dimensional.

\item On an arbitrary topological space $X$ with an arbitrary Abelian sheaf $\fF$, Čech cohomology may not give the same result as the derived functor cohomology.
But here we show that $H^1$, there is an isomorphism if one takes the limit over all coverings.
\begin{itemize}
  \item[(a)] Let $\goth{U} = (U_i)_{i \in I}$ be an open covering of the topological space $X$.
  A \textit{refinement} of $\goth{U}$ is a covering $\goth{B} = (V_j)_{j \in J}$, together with a map $\lambda : J \to I$ of the index sets, such that for each $j \in J$, $V_j \subseteq U_{\lambda(j)}$.
  If $\goth{B}$ is a refinement of $\goth{U}$, show that there is a natural induced map on Čech cohomology for any Abelian sheaf $\fF$, and for each $i$,
  \begin{equation*}
    \lambda^i : \check{H}^i(\goth{U}, \fF) \to \check{H}^i(\goth{B}, \fF).
  \end{equation*}
  The coverings of $X$ form a partially ordered set under refinement, so we can consider the Čech cohomology in the limit
  \begin{equation*}
    \varinjlim_\goth{U} = \check{H}^i(\goth{U}, \fF).
  \end{equation*}

  \item[(b)]  For any Abelian sheaf $\fF$ on $X$, show that the natural maps (4.4) for each covering
  \begin{equation*}
    \check{H}^i(\goth{U}, \fF) \to H^i(X, \fF)
  \end{equation*}
  are compatible with the refinement maps above.

  \item[(c)] Now prove the following theorem.
  Let $X$ be a topological space, $\fF$ a sheaf of Abelian groups.
  Then the natural map
  \begin{equation*}
    \varinjlim_\goth{U} \check{H}^1(\goth{U}, \fF) \to H^1(X, \fF)
  \end{equation*}
  is an isomorphism.
\end{itemize}

\item For any ringed space $(X, \fO_X)$, let $\pic{X}$ be the group of isomorphism classes of invertible sheaves (II, \S 6).
Show that $\pic{X} \cong H^1(X, \fO_X^*)$ where $\fO_X^*$ denotes the sheaf whose sections over an open set $U$ are the units in the ring $\Gamma(U, \fO_X)$, with multiplication as the group operation.

\end{enumerate}
\end{document}

% \item Let $(X, \fO_X)$ be a ringed space, let $\fI$ be a sheaf of ideals with $\fI^2 = 0$, and let $X_0$ be the ringed space $(X, \fO_X / \fI)$. Show that there is an exact sequence of sheaves of Abelian groups on $X$,
% \begin{equation*}
%   0 \to \mathscr{J} \to \fO_X^* \to \fO_{X_0}^* \to 0,
% \end{equation*}
% where $\fO_X^*$ (respectively, $\fO_{X_0}^*$) denotes the sheaf of (multiplicative) groups of units in the sheaf of rings $\fO_X$ (respectively, $\fO_{X_0}$); the map $\fI \to \fO_X^*$ is defined by $a \mapsto 1 + a$, and $\fI$ has its usual (additive) group structure. Conclude there is an exact sequence of Abelian groups
% \begin{equation*}
%   \cdots \to H^1(X, \fI) \to \pic{X} \to \pic{X_0} \to H^2(X, \fJ) \to \cdots.
% \end{equation*}

% \item Let $X$ be a subscheme of $\PP_k^2$ defined by a single homogenous equation $f(x_0, x_1, x_2) = 0$ of degree $d$. (Do no assume $f$ is irreducible.) Assume that $(1, 0, 0)$ is not on $X$. Then show that $X$ can be covered by the two open affine subsets $U = X \cap \{ x_1 \neq 0 \}$ and $V = X \cap \{ x_2 \neq 0 \}$. Now calculate the Čech complex
% \begin{equation*}
%   \Gamma(U, \fO_X) \oplus \Gamma(V, \fO_X) \to \Gamma(U \cap V, \fO_X)
% \end{equation*}
% explicitly, and thus show that
% \begin{align*}
%   \dim H^0(X, \fO_X) & = 1, \\
%   \dim H^1(X, \fO_X) & = \frac{1}{2}(d - 1)(d - 2).
% \end{align*}

% \item \textit{Cohomological Dimension.} Let $X$ be a noetherian separated scheme. We define the \textit{cohomological dimension} of $X$, denoted $\cd(X)$, to be the least integer $n$ such that $H^i(X, \fF) = 0$ for all quasi-coherent sheaves $\fF$ and all $i > n$. Thus, for example, Serre's theorem (3.7) says that $\text{cd}(X) = 0$ if and only if $X$ is affine. Grothendieck's theorem (2.7) implies that $\text{cd}(X) \leq \text{dim}~X$.
% \begin{itemize}
%   \item[(a)] In the definition of $\text{cd}(X)$, show that it is sufficient to consider only coherent sheaves on $X$. Use (II, Ex. 5.15) and (2.9).
%   \item[(b)] If $X$ is quasi-projective over a field $k$, then it is even sufficient to consider only locally free coherent sheaevs on $X$. Use (II, 5.18).
%   \item[(c)] Suppose $X$ has a covering by $r + 1$ open affine subsets. Use Čech cohomology to show that $\text{cd}(X) \leq r$.
%   \item[(d)] If $X$ is a quasi-projective scheme of dimsenion $r$ over a field $k$, then $X$ can be covered by $r + 1$ open affine subsets. Conclude (independently of (2.7)) that $\text{cd}(X) \leq \dim{X}$.
%   \item[(e)] Let $Y$ be a set-theoretic complete intersection (I, Ex. 2.17) of codimension $r$ in $X = \PP_k^n$. Show that $\text{cd}(X - Y) \leq r - 1$.
% \end{itemize}

% \item Let $X = \spec{k[x_1, x_2, x_3, x_4]}$ be affine four-space over a field $k$. Let $Y_1$ be the plane $x_1 = x_2 = 0$ and let $Y_2$ be the plane $x_3 = x_4 = 0$. Show that $Y = Y_1 \cup Y_2$ is not a set-theoretic complete intersection in $X$. Therefore, the projective closure $\overline{Y}$ in $\PP_k^4$ is also not a set-theoretic complete intersection.

% \item Let $X$ be a nonsingular variety over an algebraically closed field $k$, and let $\fF$ be a coherent sheaf on $X$. Show that there is a one-to-one correspondence between the set of infinitesimal extensions of $X$ by $\fF$ (II, Ex. 8.7) up to isomorphism, and the group $H^1(X, \fF \otimes \mathscr{T})$, where $\mathscr{T}$ is the tangent sheaf of $X$ (II, \S 8).

% \item This exercise shows that Čech cohomology will agree with the usual cohomology whenever the sheaf has no cohomology on any of the open sets. More precisely, let $X$ be a topological space, $\fF$ a sheaf of Abelian groups, and $\goth{U} = (U_i)$ an open cover. Assume for any finite intersection $V = U_{i_0} \cap \cdots \cap U_{i_p}$ of open sets of the covering, and for any $k > 0$, that $H^i(V, \restr{\fF}{V}) = 0$. Then prove that for all $p \geq 0$, the natural maps $$\check{H}^p(\goth{U}, \fF) \to H^p(X, \fF)$$ of (4.4) are isomorphisms. Show also that one can recover (4.5) as a corollary of this more general result.
