\documentclass{article}
\usepackage[margin=0.75in]{geometry}
\usepackage{amsmath}
\usepackage{amsthm}
\usepackage{amssymb}
\usepackage{enumitem}
\usepackage{tikz-cd}
\usepackage{yfonts}
\usepackage{mathrsfs}
\DeclareMathAlphabet{\mathpzc}{OT1}{pzc}{m}{it}
\newcommand{\goth}[1]{\mathfrak{#1}}
\newcommand{\fF}{\mathscr{F}}
\newcommand{\fG}{\mathscr{G}}
\newcommand{\fE}{\mathscr{E}}
\newcommand{\fO}{\mathscr{O}}
\newcommand{\fL}{\mathscr{L}}
\newcommand{\fM}{\mathscr{M}}
\newcommand{\fI}{\mathscr{I}}
\newcommand{\fT}{\mathscr{T}}
\newcommand{\fK}{\mathscr{K}}
\newcommand{\fS}{\mathscr{S}}
\newcommand{\fJ}{\mathscr{J}}
\newcommand{\fR}{\mathscr{R}}
\newcommand{\fH}{\mathscr{H}}
\newcommand{\fN}{\mathscr{N}}
\newcommand{\PP}{\mathbb{P}}
\newcommand{\gm}{\goth{m}}
\newcommand{\A}{\mathbb{A}}
\newcommand{\R}{\mathbb{R}}
\newcommand{\C}{\mathbb{C}}
\newcommand{\Q}{\mathbb{Q}}
\newcommand{\N}{\mathbb{N}}
\newcommand{\Z}{\mathbb{Z}}
\newtheorem{theorem}{Theorem}
\newtheorem{lemma}{Lemma}
\newtheorem{corollary}{Corollary}
\DeclareMathOperator{\id}{id}
\DeclareMathOperator{\bProj}{\mathpzc{Proj}}
\DeclareMathOperator{\Frac}{Frac}
\DeclareMathOperator{\rk}{rank}
\DeclareMathOperator{\pic}{Pic}
\DeclareMathOperator{\cacl}{CaCl}
\DeclareMathOperator{\trd}{tr.d.}
\DeclareMathOperator{\cl}{Cl}
\DeclareMathOperator{\Div}{Div}
\DeclareMathOperator{\coker}{coker}
\DeclareMathOperator{\len}{length}
\DeclareMathOperator{\height}{height}
\DeclareMathOperator{\supp}{Supp}
\DeclareMathOperator{\proj}{Proj}
\DeclareMathOperator{\im}{im}
\DeclareMathOperator{\Hom}{Hom}
\DeclareMathOperator{\Der}{Der}
\DeclareMathOperator{\spec}{Spec}
\DeclareMathOperator{\rHom}{\mathpzc{Hom}}
\newcommand\srestr[2]{{
  \left.\kern-\nulldelimiterspace % automatically resize the bar with \right
  #1 % the function
  \vphantom{\small|} % pretend it's a little taller at normal size
  \right|_{#2} % this is the delimiter
}}
\newcommand\restr[2]{{% we make the whole thing an ordinary symbol
  \left.\kern-\nulldelimiterspace % automatically resize the bar with \right
  #1 % the function
  \vphantom{\big|} % pretend it's a little taller at normal size
  \right|_{#2} % this is the delimiter
}}

% 2.4, 2.6, 2.7
% 3.1, 3.2, 3.6, 3.7
% 4.1, 4.2, 4.3, 4.4, 4.5
% 5.1, 5.2, 5.3, 5.10
% 6.1, 6.3, 6.6, 6.7
% 7.1, 7.3
% 8.1, 8.2, 8.3, 
% 9.3, 9.4, 9.11
% 10.1, 10.2, 10.3, 10.5, 10.6
% 11.1, 11.2, 11.8
% 12.1, 12.2

\title{Chapter 3, Section 11}

\usepackage{xcolor}

\pagecolor[RGB]{8,27,31}

\color[RGB]{255,255,255}

\begin{document}
\maketitle
\begin{enumerate} [label=\textbf{\arabic*.}, leftmargin=0em]

\item Show that the result of (11.2) is false without the projective hypothesis. For example, let $X = \A_k^n$, let $P = (0, \dots, 0)$, let $U = X - P$, and let $f: U \to X$ be the inclusion. Then the fibers of $f$ all have dimension $0$, but $R^{n - 1} f_* \fO_U \neq 0$.

\item Show that a projective morphism with finite fibers (= quasi-finite (II, Ex. 3.5)) is a finite morphism.

\item Let $X$ be a normal, projective variety over an algebraically closed  field $k$. Let $\goth{d}$ be a linear system (of effective Cartier divisors) without base points, and assume that $\goth{d}$ is \textit{not composite with a pencil}, which means that if $f : X \to \PP_k^n$ is the morphism determined by $\goth{d}$, then $\dim{f(X)} \geq 2$. Then show that every divisor in $\goth{d}$ is connected. This improves Bertini's theorem (10.9.1).

\item \textit{Principle of Connectedness.} Let $\{ X_t \}$ be a flat family of closed subschemes of $\PP_k^n$ parameterized by an irreducible curve $T$ of finite type over $k$> Suppose there is a nonempty open set $Y \subseteq T$, such that for all closed points $t \in Y$, $X_t$ is connected. Then prove that $X_t$ is connected for all $t \in T$.

\item Let $Y$ be a hypersurface in $X = \PP_k^N$ with $N \geq 4$. Let $\hat{X}$ be the formal completion of $X$ along $Y$ (II, \S 9). Prove that the natural map $\pic{\hat{X}} \to \pic{Y}$ is an isomorphism.

\item Again let $T$ be a hypersurface in $X = \PP_k^N$, this time with $N \geq 2$.
\begin{itemize}
  \item[(a)] If $\fF$ is a locally free sheaf on $X$, show that the natural map
  \begin{equation*}
    H^0(X, \fF) \to H^0(\hat{X}, \hat{\fF})
  \end{equation*}
  is an isomorphism.
  \item[(b)] Show that the following conditions are equivalent:
  \begin{itemize}
    \item[(i)] for each locally free sheaf $\goth{F}$ on $\hat{X}$, there exists a coherent sheaf $\fF$ on $X$ such that $\goth{F} \cong \hat{\fF}$ (i.e., $\goth{F}$ is \textit{algebraizable});
    \item[(ii)] for each locally free sheaf $\goth{F}$ on $\hat{X}$, there is an integer $n_0$ such that $\goth{F}(n)$ is generated by global sections for all $n \geq n_0$.
  \end{itemize}
  \item[(c)] Show that the conditions (i) and (ii) of (b) imply that the natural map $\pic{X} \to \pic{\hat{X}}$ is an isomorphism.

  \textit{Note.} In fact, (i) and (ii) always hold if $N \geq 3$. This fact, coupled with (Ex. 11.5) leads to Grothendieck's proof of the Lefschetz theorem which says that if $Y$ is a hypersurface in $\PP_k^N$ with $N \geq 4$, then $\pic{Y} \cong \Z$, and it is generated by $\fO_Y(1)$.
\end{itemize}

\item Now let $Y$ be a curve in $X = \PP_k^2$.
\begin{itemize}
  \item[(a)] Use the method of (Ex. 11.5) to show that $\pic{\hat{X}} \to \pic{Y}$ is surjective, and its kernel is an infinite-dimensional vector space over $k$.
  \item[(b)] Conclude that there is an invertible sheaf $\goth{L}$ on $\hat{X}$ which is not algebraizable.
  \item[(c)] Conclude also that there is a locally free sheaf $\goth{F}$ on $\hat{X}$ so that no twist $\goth{F}(n)$ is generated by global sections.
\end{itemize}

\item Let $f : X \to Y$ be a projective morphism, let $\fF$ be a coherent sheaf on $X$ which is flat over $Y$, and assume that $H^i(X_y, \fF_y) = 0$ for some $i$ and some $y \in Y$. Then show that $R^if_*(\fF)$ is $0$ in a neighborhood of $y$.

\end{enumerate}
\end{document}
