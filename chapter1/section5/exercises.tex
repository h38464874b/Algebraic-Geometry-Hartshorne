\documentclass{article}
\usepackage[margin=1in]{geometry}
\usepackage{amsmath,amsfonts,mathtools,amsthm,amssymb}
\usepackage[l2tabu,orthodox]{nag}
\usepackage{microtype} % fixes spacing or whatever
\usepackage{enumitem}
\usepackage{tikz-cd}
\usepackage{xcolor}
\usepackage{physics}
\usepackage{mathrsfs}
% \usepackage{kpfonts}
% \usepackage[cal=dutchcal,
% bb=boondox,
% bbscaled=1.05,
% scr=boondoxupr]{mathalpha}

\renewcommand{\epsilon}{\varepsilon}
\renewcommand{\phi}{\varphi}

\DeclareMathAlphabet{\mathpzc}{OT1}{pzc}{m}{it}

\newcommand{\goth}[1]{\mathfrak{#1}}
\newcommand{\red}[1]{#1_{\text{red}}}
\newcommand{\reg}[1]{#1_{\text{reg}}}

\newcommand{\fA}{\mathscr{A}}
\newcommand{\fE}{\mathscr{E}}
\newcommand{\fF}{\mathscr{F}}
\newcommand{\fG}{\mathscr{G}}
\newcommand{\fH}{\mathscr{H}}
\newcommand{\fI}{\mathscr{I}}
\newcommand{\fJ}{\mathscr{J}}
\newcommand{\fK}{\mathscr{K}}
\newcommand{\fL}{\mathscr{L}}
\newcommand{\fM}{\mathscr{M}}
\newcommand{\fN}{\mathscr{N}}
\newcommand{\fO}{\mathscr{O}}
\newcommand{\fP}{\mathscr{P}}
\newcommand{\fR}{\mathscr{R}}
\newcommand{\fS}{\mathscr{S}}
\newcommand{\fT}{\mathscr{T}}
\newcommand{\fX}{\mathscr{X}}
\newcommand{\fY}{\mathscr{Y}}
\newcommand{\fZ}{\mathscr{Z}}

\newcommand{\gm}{\goth{m}}
\newcommand{\gF}{\goth{F}}
\newcommand{\gU}{\goth{U}}
\newcommand{\gV}{\goth{V}}

\newcommand{\A}{\mathbf{A}}
\newcommand{\C}{\mathbf{C}}
\newcommand{\F}{\mathbf{F}}
\newcommand{\G}{\mathbf{G}}
\newcommand{\N}{\mathbf{N}}
\newcommand{\PP}{\mathbf{P}}
\newcommand{\Q}{\mathbf{Q}}
\newcommand{\R}{\mathbf{R}}
\newcommand{\Z}{\mathbf{Z}}

\newcommand\srestr[2]{{\left.\kern-\nulldelimiterspace #1\vphantom{\small|} \right|_{#2}}}
\newcommand\restr[2]{{\left.\kern-\nulldelimiterspace #1 \vphantom{\big|} \right|_{#2}}}

\newtheorem{theorem}{Theorem}
\newtheorem{lemma}{Lemma}
\newtheorem{corollary}{Corollary}
\newtheorem{proposition}{Proposition}

\DeclareMathOperator{\rProj}{\mathpzc{Proj}}
\DeclareMathOperator{\rSpec}{\mathpzc{Spec}}
\DeclareMathOperator{\rHom}{\mathpzc{Hom}}
\DeclareMathOperator{\id}{id}
\DeclareMathOperator{\Frac}{Frac}
\DeclareMathOperator{\rk}{rank}
\DeclareMathOperator{\pic}{Pic}
\DeclareMathOperator{\cacl}{CaCl}
\DeclareMathOperator{\trd}{tr.d.}
\DeclareMathOperator{\cl}{Cl}
\DeclareMathOperator{\depth}{depth}
\DeclareMathOperator{\codim}{codim}
\DeclareMathOperator{\Div}{Div}
\DeclareMathOperator{\coker}{coker}
\DeclareMathOperator{\len}{length}
\DeclareMathOperator{\height}{height}
\DeclareMathOperator{\supp}{Supp}
\DeclareMathOperator{\proj}{Proj}
\DeclareMathOperator{\im}{im}
\DeclareMathOperator{\Hom}{Hom}
\DeclareMathOperator{\Der}{Der}
\DeclareMathOperator{\spec}{Spec}
\DeclareMathOperator{\Aut}{Aut}
\DeclareMathOperator{\ch}{char}
\DeclareMathOperator{\tor}{Tor}
\DeclareMathOperator{\Ann}{Ann}
\DeclareMathOperator{\Syl}{Syl}
\DeclareMathOperator{\Sym}{Sym}
\DeclareMathOperator{\GL}{GL}
\DeclareMathOperator{\SL}{SL}
\DeclareMathOperator{\Stab}{Stab}
\DeclareMathOperator{\Perm}{Perm}
\DeclareMathOperator{\Orb}{Orb}
\DeclareMathOperator{\Gal}{Gal}
\DeclareMathOperator{\Supp}{Supp}

\author{James Lee}

% Solarized
% \pagecolor[RGB]{0,20,26}
% \color[RGB]{191,191,191}

% Sephia
\pagecolor[RGB]{249,239,220}

% Black
% \pagecolor[RGB]{20,20,20}
% \color[RGB]{200,200,200}

\title{Chapter 1, Section 5}

\begin{document}
\maketitle
\begin{enumerate} [label=\textbf{\arabic*.}, leftmargin=0cm]

\item[\textbf{1.}] Locate the singular points and sketch the following curves in $\A^2$ (assume $\text{char}~k \neq 2$):
\begin{itemize}
    \item[(a)] $x^2 = x^4 + y^4$;
    \item[(b)] $xy = x^6 + y^6$;
    \item[(c)] $x^3 = y^2 + x^4 + y^4$;
    \item[(d)] $x^2y + xy^2 = x^4 + y^4$.
\end{itemize}

\begin{proof} $ $ \vspace{0pt}
    \begin{center}
        \renewcommand{\arraystretch}{1.5}
        \begin{tabular}{c|c|c|c|c}
             $f$ & $\partial f / \partial x$ & $\partial f / \partial y$ & $\langle \partial f / \partial x, \partial f / \partial y\rangle = \langle 0, 0\rangle$ & $\text{Sing}~f$ \\ \hline
            $x^4 - x^2 + y^4$ & $4x^3 - 2x$ & $4y^3$ & $(0, 0)$, $(\pm 1/\sqrt{2}, 0)$ & $(0, 0)$ \\ \hline
            $x^6 + y^6 - xy$ & $6x^5 - y$ & $6y^5 - x$ & $(0, 0)$, $(6^{-1/4}, 6^{-1/4})$ & $(0, 0)$ \\ \hline
            $x^4 + y^4 - x^3 + y^2$ & $4x^3 - 3x^2$ & $4y^3 - 2y$ & $(0, 0), (3/4, 0), (0, \pm 1 / \sqrt{2}), (3/4, \pm 1 / \sqrt{2})$ & $(0, 0)$ \\ \hline
            $x^4 + y^4 - x^2 y - xy^2$ & $4x^3 - 2xy - y^2$ & $4y^3 - 2xy - x^2$ & $(0, 0)$, $(3/4, 3/4)$ & $(0, 0)$
        \end{tabular}
    \end{center}
    (a) is the tacnode, (b) is the node, (c) is the cusp, and (d) is the triple point.
\end{proof}

\item[\textbf{2.}] Locate the singular points and describe the singularities of the following surfaces in $\A^3$ (assume $\text{char}~k \neq 2$).
\begin{itemize}
    \item[(a)] $xy^2 = z^2$;
    \item[(b)] $x^2 + y^2 = z^2$;
    \item[(c)] $xy + x^3 + y^3 = 0$.
\end{itemize}

\begin{proof} $ $ \vspace{0pt}
    \begin{center}
        \renewcommand{\arraystretch}{1.5}
        \begin{tabular}{c|c|c|c|c}
             $f$ & $\partial f / \partial x$ & $\partial f / \partial y$ & $\partial f / \partial z$ & $\text{Sing}~f$ \\ \hline
            $xy^2 - z^2$ & $y^2$ & $2xy$ & $-2z$ & $(t, 0, 0), ~ t \in k$ \\ \hline
            $x^2+y^2-z^2$ & $2x$ & $2y$ & $-2z$ & $(0, 0, 0)$ \\ \hline
            $x^3+y^3+xy$ & $3x^2 + y$ & $3y^2 + x$ & $0$ & $(0, 0, t), ~ t \in k$
        \end{tabular}
    \end{center}
    (a) is the pinch point, (b) is the conical double point, and (c) is the double line.
\end{proof}

\item[\textbf{3.}] \textit{Multiplicities.} Let $Y \subseteq \A^2$ be a curve defined by the equation $f(x, y) = 0$. Let $P = (a, b)$ be a point of $\A^2$. Make a linear change of coordinates so that $P$ becomes the point $(0, 0)$. Then write $f$ as a sum $f = f_0 + f_1 + \cdots + f_d$, where $f_i$ is a homogenous polynomial of degree $i$ in $x$ and $y$. Then we define the \textit{multiplicity} of $P$ on $Y$, denoted $\mu_P(Y)$, to be the least $r$ such that $f_r \neq 0$. (Note that $P \in Y$ $\iff$ $\mu_P(Y) > 0$.) The linear factors of $f_r$ are called the \textit{tangent directions at} $P$.
\begin{itemize}
    \item[(a)] Show that $\mu_P(Y) = 1$ $\iff$ $P$ is a nonsingular point of $Y$.
    \item[(b)] Find the multiplicity of each of the singular points in Exercise 1 above.
\end{itemize}

\begin{proof} $ $ \vspace{0pt}
    \begin{itemize} [leftmargin=0cm]
        \item[(a)] Let $f'$ be the polynomial $f(x, y)$ after change of coordinates so that $P$ becomes the origin and let $O = (0, 0)$ be the origin. Then, $P$ is a nonsingular point of $Y$ $\iff$ $(\partial f / \partial x)(P), ~ (\partial f / \partial y)(P) \neq 0$ $\iff$ $(\partial f' / \partial x)(O), ~ (\partial f' / \partial y)(O) \neq 0$ $\iff$ $\partial f' / \partial x, ~ \partial f' / \partial y = \text{higher degree terms} + \text{constant}$ $\iff$ $f'$ has linear terms $\iff$ $\mu_P(Y) = 1$.

        \item[(b)] (a), (b), and (c) has multiplicity $2$, and (d) has multiplicity $3$ at $O$.
    \end{itemize}
\end{proof}

\item[\textbf{6.}] \textit{Blowing Up Curve Singularities.}
\begin{itemize}
    \item[(a)] Let $Y$ be the cusp or node of Exercise 1.
    Show that the curve $\widetilde{Y}$ obtained by blowing up $Y$ at $O = (0, 0)$ is nonsingular.

    \item[(b)] We define a \textit{node} (also called \textit{ordinary double point}) to be a double point (i.e., a point of multiplicity $2$) of a plane curve with distinct tangent directions.
    If $P$ is a node on a plane curve $Y$, show that $\varphi^{-1}(P)$ consists of two distinct nonsingular points on the blown-up curve $\widetilde{Y}$.
    We say that "blowing up $P$ resolves the singularity at $P$".

    \item[(c)] Let $P \in Y$ be the tacnode of Exercise 1.
    If $\varphi : \widetilde{Y} \to Y$ is the blowing-up at $P$, show that $\varphi^{-1}(P)$ is a node.
    Using (b) we see that the tacnode can be resolved by two successive blowing-ups.

    \item[(d)] Let $Y$ be the plane curve $y^3 = x^5$, which has a "high order cusp" at $O$.
    Show that $O$ is a triple point; that blowing up $O$ gives rise to a double point (what kind?) and that one further blowing up resolves the singularity.
\end{itemize}

\begin{proof} $ $ \vspace{0pt}
    \begin{itemize} [leftmargin=0cm]
        \item[(a)] Let $Y$ be the curve defined by the equation $xy = x^6 + y^6$. It is the node of Exercise 1. We consider the equations $xy = x^6 + y^6$ and $xu = ty$ in $\A^2 \times \PP^1$, where $t$, $u$ are homogenous coordinates for $\PP^1$. Now $\PP^1$ is covered by open sets $t \neq 0$ and $u \neq 0$, which we can consider separately. If $t \neq 0$, we can set $t = 1$, and use $u$ as an affine parameter. Then we have the equations
        \begin{align*}
            xy & = x^6 + y^6 \\
            y & = xu
        \end{align*}
        in $\A^3$ with coordinates $x$, $y$, $u$. Substituting, we get $x^2 u = x^6 + x^6 u^6$, which factors. Thus, we obtain two irreducible components, one defined by $x = 0$, $y = 0$, $u$ arbitrary, which is the exceptional curve $E$, and the other defined by $u = x^4 + x^4 u^6$. This is $\widetilde{Y}$, which meets $E$ at the point $u = 0$. This curve is nonsingular since if we set $f(x, y, u) = x^4(1 + u^6) - u$, then
        \begin{equation*}
            \pdv{f}{x}(O) = 0, \quad \pdv{f}{y}(O) = 0, \quad \pdv{f}{u}(O) = -1,
        \end{equation*}
        so $\widetilde{Y}$ is nonsingular for all points $t \neq 0$. The curve $Y$ is symmetric with respect to $x = y$, so $\widetilde{Y}$ is also nonsingular for all points $u \neq 0$, hence $\widetilde{Y}$ is nonsingular.

        \item[(b)] We can assume $P$ is the origin so that $Y$ is defined by $f = f_2 + f_3 + \cdots + f_r$. Thus, we can write $f_2 = (ax + by)(cx + dy)$, where $(a, b) \neq (c, d)$ as points in $\PP^1$. Take $t$, $u$ to be homogenous coordinates for $\PP^1$. If $t \neq 0$, substituting $y = xu$ we have the equation
        \begin{equation*}
            F_u = (a + bu)(c + du) + f_3' + \cdots + f_r' = 0,
        \end{equation*}
        where $f'_i$ is obtained by substituting $y = xu$ into $f_i$ and dividing by $x^2$, which is possible since $f_i$ is homogenous of degree $i > 2$. This equation defines $\widetilde{Y}$, which meets the exceptional curve at $(b, -a)$ and $(d, -c)$. Similarly, if $u \neq 0$, then we obtain the equation
        \begin{equation*}
            F_t = (at + b)(ct + d) + \hat{f}_3 + \cdots + \hat{f}_r = 0,
        \end{equation*}
        thus $(b, -a)$ and $(d, -c)$ are the only points $\widetilde{Y}$ meets the exceptional curve, that is $\varphi^{-1}(P) = \{ (a, b), (c, d) \}$. Also, $\widetilde{Y}$ is nonsingular at both points since
        \begin{equation*}
            \pdv{F_u}{u}(0, 0, - a / b) = b\bigg( c - \frac{ad}{b} \bigg) \neq 0, \quad \pdv{F_t}{u}(0, 0, - c / d) = b\bigg( a - \frac{bc}{d} \bigg) \neq 0.
        \end{equation*}

        \item[(c)] Let $Y$ be the plane curve defined by $x^2 = x^4 + y^4$ and let $t$, $u$, be homogenous coordinates for $\PP^1$. If $t \neq 0$, then we have the equations
        \begin{align*}
            x^2 & = x^4 + y^4 \\
            y & = xu
        \end{align*}
        in $\A^3$ with coordinates $x$, $y$, $u$. Substituting, we get $x^2 = x^4 + x^4 u^4$, thus $\widetilde{Y}$ is defined by $1 = x^2 + x^2u^4$, which intersects the exceptional curve at $u = 0$. Similarly, if $ u \neq 0$, then $\widetilde{Y}$ is defined by $t^2 = t^2 y^2 + y^2$ and intersects the exceptional curve at $t = 0$. These two points correspond to the slopes of the two branches of $Y$ at $P$, which are the two distinct tangent directions on the node $P$.

        \item[(d)] If $t \neq 0$, then we have the equation $x^2 = u^3$, which is a cusp.
        If $u \neq 0$, then we have the equation $1 = t^5 y^2$, which does not meet the exceptional curve.
    \end{itemize}
\end{proof}

\item[\textbf{8.}] Let $Y \subseteq \PP^n$ be a projective variety of dimension $r$. Let $f_1, \dots, f_t \in S = k[x_0, \dots, x_n]$ be homogenous polynomials which generate the ideal of $Y$. Let $P \in Y$ be a point, with homogenous coordinates $P = (a_0, \dots, a_n)$. Show that $P$ is nonsingular on $Y$ if and only if the rank of the matrix $\norm{(\partial f_i / \partial x_j)(a_0, \dots, a_n)}$ is $n - r$.

\begin{proof}
    We follow the hint.
    The matrix $\norm{(\partial f_i / \partial x_i)(a_0, \dots, a_n)}$ has rank $n-r$ if and only if there exists a $(n - r) \times (n - r)$ submatrix with nonzero determinant.
    The determinant of a matrix of homogenous polynomials where entries of the same row have same degree is also a homogenous polynomial, hence the rank of $\norm{(\partial f_i / \partial x_j)(a_0, \dots, a_n)}$ is independent of the choice of homogenous coordinates of $P$.
    Assuming $a_0 \neq 0$, we can pass to an open affine $U_0 \subseteq \PP^n$ containing $P$ and use the affine Jacobian matrix, where $U_0$ is the open affine subset consisting of all points with nonzero $0$th coordinate.
    In particular, f we set $Y_0 = Y \cap U_0$, then $g_i(y_1, \dots, y_n) = f_i(1, y_1, \dots, y_n)$ ($1 \leq i \leq t$) generate the ideal of $Y_0$.
    Thus, $P$ is nonsingular if and only if the rank of the matrix $\norm{(\partial g_i / \partial y_j)(a_1 / a_0, \dots, a_n / a_0)}$ is $n - r$.
    Then, we have
    \begin{equation*}
        \pdv{g_i}{y_j} = \pdv{f_i(1, y_1, \dots, y_n)}{y_j} = \pdv{f_i(1, x_1, \dots, x_n)}{x_j},
    \end{equation*}
    so $|| (\partial g_i / \partial y_j)(a_1 / a_0, \dots, a_n / a_0) ||$ has an invertible $(n - r) \times (n - r)$ submatrix if and only if $\norm{(\partial f_i / \partial x_j)(a_0, \dots, a_n)}$ has an invertible $(n - r) \times (n - r)$ submatrix since the rank of a matrix is independent of the homogenous coordinates chosen for $P$.
\end{proof}

\item[\textbf{10.}] For a point $P$ on a variety $X$, let $\goth{m}$ be the maximal ideal of the local ring $\fO_P$. We define the \textit{Zariski tangent space} $T_P(X)$ of $X$ at $P$ to be the dual $k$-vector space of $\goth{m} / \goth{m}^2$.
\begin{itemize}
    \item[(a)] For any point $P \in X$, $\dim{T_P(X)} \geq \dim{X}$, with equality if and only if $P$ is nonsingular.
    \item[(b)] For any morphism $\varphi : X \to Y$, there is a natural induced $k$-linear map $T_P(\varphi) : T_P(X) \to T_{\varphi(P)}(Y)$.
    \item[(c)] If $\varphi$ is the vertical projection of the parabola $x = y^2$ onto the $x$-axis, show that the induced map $T_0(\varphi)$ of tangent spaces at the origin is the zero map.
\end{itemize}

\begin{proof} $ $ \vspace{0pt}
    \begin{itemize} [leftmargin=0cm]
        \item[(a)] Since $\goth{m} / \goth{m}^2$ is a finite dimension $k$-vector space, $\dim_k{T_p(X)} = \dim_k{\goth{m} / \goth{m}^2} \geq \dim{X}$ by (5.2A), and $\dim_k{\goth{m} / \goth{m}^2} = \dim{\fO_{P, X}} = \dim{X}$ by definition of nonsingular points and Exercise 3.12.

        \item[(b)] A morphism $\varphi : X \to Y$ induces a map $\varphi_{P}^* : \fO_{\varphi(P), Y} \to \fO_{P, X}$ between local rings, where the maximal ideal $\goth{n}$ of $\fO_{\varphi(P), Y}$ is mapped into the maximal ideal $\goth{m}$ of $\fO_{P, X}$. Thus, we can define $T_P(\varphi)F$ as
        \begin{equation*}
            (T_P(\varphi)F)f = F(d_\varphi f) = F(f \circ \varphi),
        \end{equation*}
        where $d_\varphi : \goth{n} / \goth{n}^2 \to \goth{m} / \goth{m}^2$ is the map induced by $\varphi_P^*$. It is clearly $k$-linear.

        \item[(c)] By the formula above, it suffices to show $d_\varphi$ is the zero map at $0$. Let $X$ be the parabola $x = y^2$ and let $Y$ by the $x$-axis. The maximal ideal $\goth{m}$ of the local ring $\fO_{0, X}$ is generated by $y$, thus $\goth{m} / \goth{m}^2$ is a one dimensional $k$-vector space spanned by $y$, and similarly $\goth{n} / \goth{n}^2$ of $\fO_{0, Y}$ is a one dimensional $k$-vector space spanned by $x$. Thus, we have $\varphi^* x = y^2 \in \goth{m}^2$, hence $d_\varphi = 0$, hence $T_0(\varphi) = 0$.
    \end{itemize} 
\end{proof}

\end{enumerate}

\end{document}

% \item \textit{Analytically Isomorphic Singularities.}
% \begin{itemize}
%     \item[(a)] If $P \in Y$ and $Q \in Z$ are analytically isomorphic plane curve singularities, show that the multiplicities $\mu_P(Y)$ and $\mu_Q(Z)$ are the same.
%     \item[(b)] Generalize the example in the text (5.6.3) to show that if $f = f_r + f_{r + 1} + \cdots \in k[[x, y]]$, and if the leading form $f_r$ of $f$ factors as $f_r = g_s h_t$, where $g_s$, $h_t$ are homogenous of degrees $s$ and $t$ respectively, and have no common linear factor, then there are formal power series
%     \begin{gather*}
%         g = g_s + g_{s + 1} + \cdots \\
%         h = h_t + h_{t + 1} + \cdots
%     \end{gather*}
%     in $k[[x, y]]$ such that $f = gh$.
%     \item[(c)] Let $Y$ be defined by the equation $f(x, y) = 0$ in $\A^2$, and let $P = (0, 0)$ be a point of multiplicity $r$ on $Y$, so that when $f$ is expanded as a polynomial in $x$ and $y$, we have $f = f_r + \text{higher terms}$. We say that $P$ is an \textit{ordinary $r$-fold} point if $f_r$ is a product of $r$ \textit{distinct} linear factors. Show that any two ordinary double points are analytically isomorphic. Ditto for ordinary triple points. But show that there is a one-parameter family of mutually nonisomorphic ordinary $4$-fold points.
%     \item[(d)] Assume $\text{char}~k \neq 2$. Show that any double point of a plane curve is analytically isomorphic to the singularity at $(0, 0)$ of the curve $y^2 = x^r$, for a uniquely determined $r \geq 2$. If $r = 2$ it is a node. If $r = 3$ we call it a \textit{cusp}; if $r = 4$ a \textit{tacnode}.
% \end{itemize}
